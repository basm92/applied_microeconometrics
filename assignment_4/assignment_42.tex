% Options for packages loaded elsewhere
\PassOptionsToPackage{unicode}{hyperref}
\PassOptionsToPackage{hyphens}{url}
%
\documentclass[
]{article}
\usepackage{lmodern}
\usepackage{amssymb,amsmath}
\usepackage{ifxetex,ifluatex}
\ifnum 0\ifxetex 1\fi\ifluatex 1\fi=0 % if pdftex
  \usepackage[T1]{fontenc}
  \usepackage[utf8]{inputenc}
  \usepackage{textcomp} % provide euro and other symbols
\else % if luatex or xetex
  \usepackage{unicode-math}
  \defaultfontfeatures{Scale=MatchLowercase}
  \defaultfontfeatures[\rmfamily]{Ligatures=TeX,Scale=1}
\fi
% Use upquote if available, for straight quotes in verbatim environments
\IfFileExists{upquote.sty}{\usepackage{upquote}}{}
\IfFileExists{microtype.sty}{% use microtype if available
  \usepackage[]{microtype}
  \UseMicrotypeSet[protrusion]{basicmath} % disable protrusion for tt fonts
}{}
\makeatletter
\@ifundefined{KOMAClassName}{% if non-KOMA class
  \IfFileExists{parskip.sty}{%
    \usepackage{parskip}
  }{% else
    \setlength{\parindent}{0pt}
    \setlength{\parskip}{6pt plus 2pt minus 1pt}}
}{% if KOMA class
  \KOMAoptions{parskip=half}}
\makeatother
\usepackage{xcolor}
\IfFileExists{xurl.sty}{\usepackage{xurl}}{} % add URL line breaks if available
\IfFileExists{bookmark.sty}{\usepackage{bookmark}}{\usepackage{hyperref}}
\hypersetup{
  pdftitle={Applied Microeconometrics - Assignment 4},
  pdfauthor={Walter Verwer (589962) \& Bas Machielsen (590049)},
  hidelinks,
  pdfcreator={LaTeX via pandoc}}
\urlstyle{same} % disable monospaced font for URLs
\usepackage[margin=1in]{geometry}
\usepackage{graphicx}
\makeatletter
\def\maxwidth{\ifdim\Gin@nat@width>\linewidth\linewidth\else\Gin@nat@width\fi}
\def\maxheight{\ifdim\Gin@nat@height>\textheight\textheight\else\Gin@nat@height\fi}
\makeatother
% Scale images if necessary, so that they will not overflow the page
% margins by default, and it is still possible to overwrite the defaults
% using explicit options in \includegraphics[width, height, ...]{}
\setkeys{Gin}{width=\maxwidth,height=\maxheight,keepaspectratio}
% Set default figure placement to htbp
\makeatletter
\def\fps@figure{htbp}
\makeatother
\setlength{\emergencystretch}{3em} % prevent overfull lines
\providecommand{\tightlist}{%
  \setlength{\itemsep}{0pt}\setlength{\parskip}{0pt}}
\setcounter{secnumdepth}{-\maxdimen} % remove section numbering
\usepackage{booktabs}
\usepackage{longtable}
\usepackage{array}
\usepackage{multirow}
\usepackage{wrapfig}
\usepackage{float}
\usepackage{colortbl}
\usepackage{pdflscape}
\usepackage{tabu}
\usepackage{threeparttable}
\usepackage{threeparttablex}
\usepackage[normalem]{ulem}
\usepackage{makecell}
\usepackage{xcolor}
\usepackage{siunitx}
\newcolumntype{d}{S[input-symbols = ()]}
\ifluatex
  \usepackage{selnolig}  % disable illegal ligatures
\fi

\title{Applied Microeconometrics - Assignment 4}
\author{Walter Verwer (589962) \& Bas Machielsen (590049)}
\date{\today}

\begin{document}
\maketitle

\begin{enumerate}
\def\labelenumi{\arabic{enumi}.}
\tightlist
\item
  Describe the sickness spell data, i.e.~do a simple listing of the
  survivor function and plot the hazard rate and the survivor function.
  Make separate plots for the first two weeks and for the first year.
  Also plot the hazard by different subgroups (for instance gender) and
  test whether the survival curves are the same for the different
  subgroups.
\end{enumerate}

First, we plot the exit rates:

\includegraphics[width=200px,height=150px]{assignment_42_files/figure-latex/unnamed-chunk-2-1}
\includegraphics[width=200px,height=150px]{assignment_42_files/figure-latex/unnamed-chunk-2-2}

\includegraphics[width=200px,height=150px]{assignment_42_files/figure-latex/unnamed-chunk-3-1}
\includegraphics[width=200px,height=150px]{assignment_42_files/figure-latex/unnamed-chunk-3-2}

Then, we plot the survivor functions:

\includegraphics[width=200px,height=150px]{assignment_42_files/figure-latex/unnamed-chunk-4-1}
\includegraphics[width=200px,height=150px]{assignment_42_files/figure-latex/unnamed-chunk-4-2}

\includegraphics[width=200px,height=150px]{assignment_42_files/figure-latex/unnamed-chunk-5-1}
\includegraphics[width=200px,height=150px]{assignment_42_files/figure-latex/unnamed-chunk-5-2}

Now we test whether the survival function and hazard rate of the
subgroups are significantly different.

\clearpage

2.1 Estimate a Weibull and an Exponential model for sickness spells.
Start with a very simple specification and you only include one
regressor and subsequently add more regressors. Comment on the change in
the Weibull parameters and the regression parameters when you add more
variables to the model. Compare the estimates of both models.

\begin{table}[!h]

\caption{\label{tab:unnamed-chunk-8}Weibull Models}
\centering
\resizebox{\linewidth}{!}{
\begin{tabular}[t]{lcccccc}
\toprule
  & Model 1 & Model 2 & Model 3 & Model 4 & Model 5 & Model 6\\
\midrule
(Intercept) & \num{1.938}*** & \num{1.795}*** & \num{2.280}*** & \num{2.279}*** & \num{2.334}*** & \num{0.979}*\\
 & (\num{0.127}) & (\num{0.195}) & (\num{0.213}) & (\num{0.213}) & (\num{0.251}) & (\num{0.530})\\
gender & \num{0.243}*** & \num{0.254}*** & \num{0.273}*** & \num{0.274}*** & \num{0.278}*** & \num{0.319}***\\
 & (\num{0.077}) & (\num{0.078}) & (\num{0.078}) & (\num{0.087}) & (\num{0.087}) & (\num{0.092})\\
Log(scale) & \num{0.490}*** & \num{0.490}*** & \num{0.487}*** & \num{0.487}*** & \num{0.486}*** & \num{0.483}***\\
 & (\num{0.013}) & (\num{0.013}) & (\num{0.013}) & (\num{0.013}) & (\num{0.013}) & (\num{0.013})\\
marstat &  & \num{0.069} & \num{0.055} & \num{0.055} & \num{0.049} & \num{0.031}\\
 &  & (\num{0.066}) & (\num{0.066}) & (\num{0.066}) & (\num{0.066}) & (\num{0.068})\\
contract &  &  & \num{-0.467}*** & \num{-0.467}*** & \num{-0.467}*** & \num{-0.423}***\\
 &  &  & (\num{0.093}) & (\num{0.093}) & (\num{0.094}) & (\num{0.093})\\
lowgroup &  &  &  & \num{-0.004} & \num{-0.005} & \num{-0.020}\\
 &  &  &  & (\num{0.079}) & (\num{0.079}) & (\num{0.080})\\
classize &  &  &  &  & \num{0.002} & \num{0.002}\\
 &  &  &  &  & (\num{0.005}) & (\num{0.005})\\
schsize &  &  &  &  & \num{0.000} & \num{0.000}\\
 &  &  &  &  & (\num{0.000}) & (\num{0.000})\\
public &  &  &  &  & \num{-0.056} & \num{-0.048}\\
 &  &  &  &  & (\num{0.091}) & (\num{0.089})\\
protest &  &  &  &  & \num{-0.153} & \num{-0.078}\\
 &  &  &  &  & (\num{0.106}) & (\num{0.108})\\
merged &  &  &  &  &  & \num{0.005}\\
 &  &  &  &  &  & (\num{0.013})\\
avgfem &  &  &  &  &  & \num{-0.206}\\
 &  &  &  &  &  & (\num{0.284})\\
avgage &  &  &  &  &  & \num{0.034}***\\
 &  &  &  &  &  & (\num{0.011})\\
avglowgr &  &  &  &  &  & \num{0.122}\\
 &  &  &  &  &  & (\num{0.259})\\
\midrule
Num.Obs. & \num{6520} & \num{6520} & \num{6520} & \num{6520} & \num{6520} & \num{6520}\\
AIC & \num{44443.9} & \num{44442.6} & \num{44415.1} & \num{44417.1} & \num{44416.8} & \num{44383.2}\\
Log.Lik. & \num{-22218.934} & \num{-22217.295} & \num{-22202.540} & \num{-22202.537} & \num{-22198.402} & \num{-22177.593}\\
\bottomrule
\multicolumn{7}{l}{\rule{0pt}{1em}* p $<$ 0.1, ** p $<$ 0.05, *** p $<$ 0.01}\\
\end{tabular}}
\end{table}

Most of the parameters remain stable over the models, with the exception
of the intercept, which has no theoretical meaning. The significant
variables are gender, the Weibull scale parameter, contract, and, in the
last specification, the average age in class. People with a less stable
contract are more likely to remain sick, and females are more likely to
get better again than males at any point in time. The average age is
also correlated positively with the hazard rate of ending a sickness
spell. \clearpage

\begin{table}[!h]

\caption{\label{tab:unnamed-chunk-10}Exponential Models}
\centering
\resizebox{\linewidth}{!}{
\begin{tabular}[t]{lcccccc}
\toprule
  & Model 1 & Model 2 & Model 3 & Model 4 & Model 5 & Model 6\\
\midrule
(Intercept) & \num{2.435}*** & \num{2.247}*** & \num{2.852}*** & \num{2.848}*** & \num{2.909}*** & \num{1.601}**\\
 & (\num{0.164}) & (\num{0.236}) & (\num{0.255}) & (\num{0.254}) & (\num{0.290}) & (\num{0.633})\\
gender & \num{0.267}*** & \num{0.280}*** & \num{0.303}*** & \num{0.324}*** & \num{0.332}*** & \num{0.377}***\\
 & (\num{0.097}) & (\num{0.098}) & (\num{0.097}) & (\num{0.104}) & (\num{0.103}) & (\num{0.108})\\
marstat &  & \num{0.091} & \num{0.077} & \num{0.076} & \num{0.077} & \num{0.058}\\
 &  & (\num{0.080}) & (\num{0.079}) & (\num{0.079}) & (\num{0.078}) & (\num{0.080})\\
contract &  &  & \num{-0.594}*** & \num{-0.595}*** & \num{-0.600}*** & \num{-0.537}***\\
 &  &  & (\num{0.112}) & (\num{0.111}) & (\num{0.112}) & (\num{0.116})\\
lowgroup &  &  &  & \num{-0.044} & \num{-0.043} & \num{-0.063}\\
 &  &  &  & (\num{0.096}) & (\num{0.097}) & (\num{0.098})\\
classize &  &  &  &  & \num{0.003} & \num{0.003}\\
 &  &  &  &  & (\num{0.005}) & (\num{0.005})\\
schsize &  &  &  &  & \num{0.000} & \num{0.000}\\
 &  &  &  &  & (\num{0.000}) & (\num{0.000})\\
public &  &  &  &  & \num{-0.116} & \num{-0.093}\\
 &  &  &  &  & (\num{0.103}) & (\num{0.104})\\
protest &  &  &  &  & \num{-0.160} & \num{-0.083}\\
 &  &  &  &  & (\num{0.124}) & (\num{0.128})\\
merged &  &  &  &  &  & \num{-0.004}\\
 &  &  &  &  &  & (\num{0.015})\\
avgfem &  &  &  &  &  & \num{-0.237}\\
 &  &  &  &  &  & (\num{0.335})\\
avgage &  &  &  &  &  & \num{0.033}***\\
 &  &  &  &  &  & (\num{0.012})\\
avglowgr &  &  &  &  &  & \num{0.091}\\
 &  &  &  &  &  & (\num{0.297})\\
\midrule
Num.Obs. & \num{6520} & \num{6520} & \num{6520} & \num{6520} & \num{6520} & \num{6520}\\
AIC & \num{49147.6} & \num{49133.1} & \num{49022.5} & \num{49022.1} & \num{48996.4} & \num{48893.8}\\
Log.Lik. & \num{-24571.812} & \num{-24563.554} & \num{-24507.235} & \num{-24506.073} & \num{-24489.220} & \num{-24433.900}\\
\bottomrule
\multicolumn{7}{l}{\rule{0pt}{1em}* p $<$ 0.1, ** p $<$ 0.05, *** p $<$ 0.01}\\
\end{tabular}}
\end{table}

The parameters of the exponential model are very similar to the
parameters of the Weibull model. The baseline hazard is, as in the
Weibull model, sensitive to the inclusion of covariates. Intuitively,
this makes sense: more covariates allows for a better isolation of the
baseline (unconditional) hazard.

\clearpage

2.2 Estimate separate Weibull models for males and females. Comment on
the results (is it better to estimate separate models for males and
females?) Estimate the Weibull duration model for other subgroups that
may differ in their behavior and where the baseline hazard may differ.

First, we estimate a model for males:

\begin{table}[!h]

\caption{\label{tab:unnamed-chunk-12}Weibull Models - Males Only}
\centering
\resizebox{\linewidth}{!}{
\begin{tabular}[t]{lcccccc}
\toprule
  & Model 1 & Model 2 & Model 3 & Model 4 & Model 5 & Model 6\\
\midrule
(Intercept) & \num{2.204}*** & \num{2.446}*** & \num{3.162}*** & \num{3.149}*** & \num{3.363}*** & \num{1.558}*\\
 & (\num{0.048}) & (\num{0.329}) & (\num{0.394}) & (\num{0.394}) & (\num{0.419}) & (\num{0.904})\\
Log(scale) & \num{0.461}*** & \num{0.460}*** & \num{0.458}*** & \num{0.457}*** & \num{0.456}*** & \num{0.449}***\\
 & (\num{0.024}) & (\num{0.024}) & (\num{0.024}) & (\num{0.024}) & (\num{0.024}) & (\num{0.024})\\
marstat &  & \num{-0.125} & \num{-0.138} & \num{-0.138} & \num{-0.150} & \num{-0.149}\\
 &  & (\num{0.168}) & (\num{0.165}) & (\num{0.166}) & (\num{0.161}) & (\num{0.157})\\
contract &  &  & \num{-0.675}*** & \num{-0.685}*** & \num{-0.631}*** & \num{-0.559}***\\
 &  &  & (\num{0.169}) & (\num{0.174}) & (\num{0.179}) & (\num{0.188})\\
lowgroup &  &  &  & \num{0.083} & \num{0.084} & \num{0.094}\\
 &  &  &  & (\num{0.122}) & (\num{0.122}) & (\num{0.123})\\
classize &  &  &  &  & \num{-0.009} & \num{-0.008}\\
 &  &  &  &  & (\num{0.007}) & (\num{0.007})\\
schsize &  &  &  &  & \num{0.000} & \num{0.000}\\
 &  &  &  &  & (\num{0.000}) & (\num{0.000})\\
public &  &  &  &  & \num{-0.073} & \num{-0.036}\\
 &  &  &  &  & (\num{0.128}) & (\num{0.132})\\
protest &  &  &  &  & \num{0.009} & \num{0.032}\\
 &  &  &  &  & (\num{0.180}) & (\num{0.174})\\
merged &  &  &  &  &  & \num{0.035}\\
 &  &  &  &  &  & (\num{0.028})\\
avgfem &  &  &  &  &  & \num{0.457}\\
 &  &  &  &  &  & (\num{0.400})\\
avgage &  &  &  &  &  & \num{0.041}**\\
 &  &  &  &  &  & (\num{0.016})\\
avglowgr &  &  &  &  &  & \num{-0.357}\\
 &  &  &  &  &  & (\num{0.410})\\
\midrule
Num.Obs. & \num{2046} & \num{2046} & \num{2046} & \num{2046} & \num{2046} & \num{2046}\\
AIC & \num{13422.3} & \num{13422.4} & \num{13416.2} & \num{13417.1} & \num{13421.4} & \num{13405.4}\\
Log.Lik. & \num{-6709.152} & \num{-6708.217} & \num{-6704.101} & \num{-6703.534} & \num{-6701.689} & \num{-6689.697}\\
\bottomrule
\multicolumn{7}{l}{\rule{0pt}{1em}* p $<$ 0.1, ** p $<$ 0.05, *** p $<$ 0.01}\\
\end{tabular}}
\end{table}

\clearpage

Next, we estimate a model for females:

\begin{table}[!h]

\caption{\label{tab:unnamed-chunk-14}Weibull Models - Females Only}
\centering
\resizebox{\linewidth}{!}{
\begin{tabular}[t]{lcccccc}
\toprule
  & Model 1 & Model 2 & Model 3 & Model 4 & Model 5 & Model 6\\
\midrule
(Intercept) & \num{2.414}*** & \num{2.228}*** & \num{2.715}*** & \num{2.755}*** & \num{2.763}*** & \num{1.729}***\\
 & (\num{0.042}) & (\num{0.138}) & (\num{0.174}) & (\num{0.197}) & (\num{0.260}) & (\num{0.643})\\
Log(scale) & \num{0.503}*** & \num{0.502}*** & \num{0.499}*** & \num{0.499}*** & \num{0.498}*** & \num{0.494}***\\
 & (\num{0.016}) & (\num{0.016}) & (\num{0.016}) & (\num{0.016}) & (\num{0.016}) & (\num{0.016})\\
marstat &  & \num{0.106} & \num{0.093} & \num{0.092} & \num{0.076} & \num{0.058}\\
 &  & (\num{0.074}) & (\num{0.074}) & (\num{0.074}) & (\num{0.073}) & (\num{0.076})\\
contract &  &  & \num{-0.435}*** & \num{-0.435}*** & \num{-0.433}*** & \num{-0.393}***\\
 &  &  & (\num{0.103}) & (\num{0.102}) & (\num{0.106}) & (\num{0.103})\\
lowgroup &  &  &  & \num{-0.048} & \num{-0.038} & \num{-0.081}\\
 &  &  &  & (\num{0.099}) & (\num{0.099}) & (\num{0.103})\\
classize &  &  &  &  & \num{0.005} & \num{0.006}\\
 &  &  &  &  & (\num{0.006}) & (\num{0.006})\\
schsize &  &  &  &  & \num{0.000} & \num{0.000}\\
 &  &  &  &  & (\num{0.000}) & (\num{0.000})\\
public &  &  &  &  & \num{-0.039} & \num{-0.030}\\
 &  &  &  &  & (\num{0.116}) & (\num{0.114})\\
protest &  &  &  &  & \num{-0.211}* & \num{-0.123}\\
 &  &  &  &  & (\num{0.127}) & (\num{0.133})\\
merged &  &  &  &  &  & \num{-0.008}\\
 &  &  &  &  &  & (\num{0.014})\\
avgfem &  &  &  &  &  & \num{-0.557}\\
 &  &  &  &  &  & (\num{0.383})\\
avgage &  &  &  &  &  & \num{0.030}**\\
 &  &  &  &  &  & (\num{0.013})\\
avglowgr &  &  &  &  &  & \num{0.403}\\
 &  &  &  &  &  & (\num{0.334})\\
\midrule
Num.Obs. & \num{4474} & \num{4474} & \num{4474} & \num{4474} & \num{4474} & \num{4474}\\
AIC & \num{31018.2} & \num{31013.9} & \num{30994.0} & \num{30995.4} & \num{30991.3} & \num{30969.4}\\
Log.Lik. & \num{-15507.109} & \num{-15503.936} & \num{-15493.017} & \num{-15492.706} & \num{-15486.663} & \num{-15471.686}\\
\bottomrule
\multicolumn{7}{l}{\rule{0pt}{1em}* p $<$ 0.1, ** p $<$ 0.05, *** p $<$ 0.01}\\
\end{tabular}}
\end{table}

The results show that the coefficients for males and females are quite
similar, but it might be different to estimate separate models because
the results do not suffer from a lack of efficiency (the same
coefficients show significance), and the log-likelihoods are smaller
(closer to zero) relative to the pooled model.

\clearpage

Now, given that under question 1 we observed a difference between the
exit rates and the survival functions for the different contract types,
we have estimated Weibull models for the contract types. However, our
results are difficult to compare. The reason for this is that the models
for the temporary and mixed contract have very little observations.
Especially, the mixed contract has too little (36 observations). For
this reason we can not make an educated comparison for this subgroup.

\begin{table}[!h]

\caption{\label{tab:unnamed-chunk-15}Weibull Models - Models for the all contract types}
\centering
\fontsize{8}{10}\selectfont
\begin{tabular}[t]{lccc}
\toprule
\multicolumn{1}{c}{ } & \multicolumn{1}{c}{Fixed contract } & \multicolumn{1}{c}{Temporary contract} & \multicolumn{1}{c}{Mixed contract} \\
\cmidrule(l{3pt}r{3pt}){2-2} \cmidrule(l{3pt}r{3pt}){3-3} \cmidrule(l{3pt}r{3pt}){4-4}
  & Model 1 & Model 2 & Model 3\\
\midrule
(Intercept) & \num{0.526} & \num{0.290} & \num{2.377}\\
 & (\num{0.523}) & (\num{1.742}) & (\num{2.100})\\
marstat & \num{0.019} & \num{0.439}** & \num{-0.323}\\
 & (\num{0.070}) & (\num{0.220}) & (\num{0.225})\\
gender & \num{0.304}*** & \num{0.781}** & \num{0.616}\\
 & (\num{0.095}) & (\num{0.309}) & (\num{0.629})\\
lowgroup & \num{-0.010} & \num{-0.083} & \num{-0.089}\\
 & (\num{0.083}) & (\num{0.271}) & (\num{0.739})\\
classize & \num{0.003} & \num{0.002} & \num{0.008}\\
 & (\num{0.005}) & (\num{0.009}) & (\num{0.035})\\
schsize & \num{0.000} & \num{0.000} & \num{0.005}*\\
 & (\num{0.000}) & (\num{0.001}) & (\num{0.003})\\
public & \num{-0.041} & \num{-0.087} & \num{-0.464}**\\
 & (\num{0.090}) & (\num{0.264}) & (\num{0.189})\\
protest & \num{-0.085} & \num{0.154} & \num{0.367}\\
 & (\num{0.108}) & (\num{0.372}) & (\num{0.522})\\
merged & \num{0.002} & \num{0.056} & \num{-0.013}\\
 & (\num{0.014}) & (\num{0.046}) & (\num{0.052})\\
avgfem & \num{-0.172} & \num{-0.796} & \num{-1.661}*\\
 & (\num{0.285}) & (\num{0.934}) & (\num{1.003})\\
avgage & \num{0.035}*** & \num{-0.013} & \num{-0.045}\\
 & (\num{0.011}) & (\num{0.038}) & (\num{0.040})\\
avglowgr & \num{0.080} & \num{0.978} & \num{0.581}\\
 & (\num{0.262}) & (\num{0.820}) & (\num{1.867})\\
Log(scale) & \num{0.490}*** & \num{0.317}*** & \num{-0.226}*\\
 & (\num{0.014}) & (\num{0.050}) & (\num{0.127})\\
\midrule
Num.Obs. & \num{6196} & \num{288} & \num{36}\\
AIC & \num{42336.4} & \num{1838.7} & \num{202.8}\\
Log.Lik. & \num{-21155.217} & \num{-906.333} & \num{-88.386}\\
\bottomrule
\multicolumn{4}{l}{\rule{0pt}{1em}* p $<$ 0.1, ** p $<$ 0.05, *** p $<$ 0.01}\\
\end{tabular}
\end{table}

\clearpage

3.1 Estimate a Piece Wise Constant (PWC) model for the entire sample.
Use the stsplit command to create multiple record data. You can have as
many steps as the data allow you to take, but first start with only a
few (3 or 4 steps). Next estimate a model with 15-20 steps, or even
more. Plot de duration pattern implied by the estimates and comment on
these and the regression parameters. How do the regression parameters
(\(\beta\)) compare with those of the Weibull model?

First, we estimate a piece wise constant model.

\includegraphics[width=200px,height=150px]{assignment_42_files/figure-latex/unnamed-chunk-16-1}
\includegraphics[width=200px,height=150px]{assignment_42_files/figure-latex/unnamed-chunk-16-2}

Above, we plot the hazard rate and survival function of our estimated
piecewise constant model. The piecewise-constant model coefficient
estimates can be found in the table below. As the attrition from the
sample is very high in the starting period, our Weibull model predicts a
high hazard rate, and thus high attrition in the first period.
Afterwards, piecewise, it attempts to catch up the imbalances by fitting
a new Weibull distribution and hazard rate (implied by the new constant
term), which significantly increases the hazard rate relative to the
previous period. This proves the model does catch up with the decay in
the data that is not as fastly decaying as implied by a Weibull model
under the previous parameters.

One other consequence of including a dummy for the first twenty days is
an underestimation of the survival probability in the early period. Even
though it mildly realistic, it underestimates survival by only focusing
on the first 20 days, under which attrition is very high. We repeat this
exercise again with a piecewise constant model with more dummies, but
omit the plot for brevity's sake.

\begin{table}[!h]

\caption{\label{tab:unnamed-chunk-18}PWC (Weibull)}
\centering
\fontsize{8}{10}\selectfont
\begin{tabular}[t]{lcc}
\toprule
\multicolumn{1}{c}{ } & \multicolumn{1}{c}{PWC 14 steps} & \multicolumn{1}{c}{PWC 4 steps} \\
\cmidrule(l{3pt}r{3pt}){2-2} \cmidrule(l{3pt}r{3pt}){3-3}
  & Model 1 & Model 2\\
\midrule
(Intercept) & \num{1.109}*** & \num{1.024}***\\
 & (\num{0.199}) & (\num{0.181})\\
gender & \num{0.056}* & \num{0.054}*\\
 & (\num{0.033}) & (\num{0.032})\\
marstat & \num{-0.032} & \num{-0.024}\\
 & (\num{0.024}) & (\num{0.023})\\
contract & \num{-0.149}*** & \num{-0.114}**\\
 & (\num{0.044}) & (\num{0.045})\\
merged & \num{0.007} & \num{0.012}*\\
 & (\num{0.007}) & (\num{0.007})\\
lowgroup & \num{0.012} & \num{0.005}\\
 & (\num{0.032}) & (\num{0.031})\\
classize & \num{-0.001} & \num{0.000}\\
 & (\num{0.001}) & (\num{0.001})\\
schsize & \num{0.000} & \num{0.000}\\
 & (\num{0.000}) & (\num{0.000})\\
public & \num{0.037} & \num{0.035}\\
 & (\num{0.038}) & (\num{0.034})\\
protest & \num{-0.005} & \num{0.001}\\
 & (\num{0.041}) & (\num{0.039})\\
avgfem & \num{0.066} & \num{0.014}\\
 & (\num{0.113}) & (\num{0.104})\\
avgage & \num{0.015}*** & \num{0.014}***\\
 & (\num{0.004}) & (\num{0.004})\\
avglowgr & \num{-0.010} & \num{0.030}\\
 & (\num{0.122}) & (\num{0.115})\\
timeperiod 10 & \num{4.632}*** & \\
 & (\num{0.246}) & \\
timeperiod 11 & \num{4.397}*** & \\
 & (\num{0.193}) & \\
timeperiod 12 & \num{4.938}*** & \\
 & (\num{0.250}) & \\
timeperiod 13 & \num{4.137}*** & \\
 & (\num{0.038}) & \\
timeperiod 14 & \num{5.551}*** & \\
 & (\num{0.305}) & \\
timeperiod 2 & \num{2.058}*** & \num{1.955}***\\
 & (\num{0.032}) & (\num{0.029})\\
timeperiod 3 & \num{2.704}*** & \num{2.967}***\\
 & (\num{0.052}) & (\num{0.048})\\
timeperiod 4 & \num{3.150}*** & \num{4.334}***\\
 & (\num{0.075}) & (\num{0.075})\\
timeperiod 5 & \num{3.412}*** & \\
 & (\num{0.080}) & \\
timeperiod 6 & \num{3.731}*** & \\
 & (\num{0.121}) & \\
timeperiod 7 & \num{4.158}*** & \\
 & (\num{0.157}) & \\
timeperiod 8 & \num{3.736}*** & \\
 & (\num{0.089}) & \\
timeperiod 9 & \num{4.550}*** & \\
 & (\num{0.268}) & \\
\midrule
Num.Obs. & \num{6473} & \num{6520}\\
AIC & \num{36079.3} & \num{36146.3}\\
Log.Lik. & \num{-18013.641} & \num{-18057.168}\\
\bottomrule
\multicolumn{3}{l}{\rule{0pt}{1em}* p $<$ 0.1, ** p $<$ 0.05, *** p $<$ 0.01}\\
\end{tabular}
\end{table}

In the table below, we show the estimates of two piece-wise constant
models. For the first model, we have plotted the hazard rate and
survival function, for the second, we omitted it for brevitiy's sake. We
see that the hazard rate is increasing over time, in general, though not
monotonically. The estimated survival curve is just the integral over
all these hazard rates, and show a very fast tendency to go to zero. As
before, this makes sense, as this is the case in the data, but piecewise
dummies focusing on a small first period make the fitted distribution
such that the survival rate is approaching zero in a very short period.

Comparing these results to the Weibull models, we observe that the
inclusion of piece-wise constants greatly affects the coefficient
estimates for the covariates: in particular, variation that was
previously attributed to the covariates is now attributed to a more
flexible baseline hazard over time: likely, we have overestimated the
influence of covariates in the preceding analyses, because the
covariates should only be modeled after the baseline hazard is
accurately specified.

\clearpage

3.2 Estimate separate models for males and females.

First we estimate a PWC model for males.

\begin{table}[!h]

\caption{\label{tab:unnamed-chunk-20}PWC (Weibull) - Males only}
\centering
\fontsize{8}{10}\selectfont
\begin{tabular}[t]{lcc}
\toprule
\multicolumn{1}{c}{ } & \multicolumn{1}{c}{PWC 14 steps} & \multicolumn{1}{c}{PWC 4 steps} \\
\cmidrule(l{3pt}r{3pt}){2-2} \cmidrule(l{3pt}r{3pt}){3-3}
  & Model 1 & Model 2\\
\midrule
(Intercept) & \num{-850.507}*** & \num{-1.016}\\
 & (\num{0.322}) & (\num{0.958})\\
marstat & \num{-0.017} & \num{0.260}**\\
 & (\num{0.060}) & (\num{0.130})\\
contract & \num{853.775}*** & \num{5.494}***\\
 & (\num{0.000}) & (\num{0.661})\\
merged & \num{0.006} & \num{0.017}\\
 & (\num{0.014}) & (\num{0.022})\\
lowgroup & \num{0.067} & \num{-0.082}\\
 & (\num{0.045}) & (\num{0.097})\\
classize & \num{0.001} & \num{-0.008}\\
 & (\num{0.003}) & (\num{0.007})\\
schsize & \num{0.000} & \num{0.000}\\
 & (\num{0.000}) & (\num{0.000})\\
public & \num{0.075} & \num{-0.103}\\
 & (\num{0.054}) & (\num{0.165})\\
protest & \num{-0.049} & \num{-0.341}**\\
 & (\num{0.052}) & (\num{0.145})\\
avgfem & \num{-0.009} & \num{-0.445}\\
 & (\num{0.190}) & (\num{0.457})\\
avgage & \num{0.003} & \num{-0.012}\\
 & (\num{0.006}) & (\num{0.015})\\
avglowgr & \num{-0.086} & \num{0.172}\\
 & (\num{0.180}) & (\num{0.338})\\
timeperiod 10 & \num{2.133}*** & \\
 & (\num{0.102}) & \\
timeperiod 11 & \num{2.191}*** & \\
 & (\num{0.105}) & \\
timeperiod 12 & \num{2.437}*** & \\
 & (\num{0.090}) & \\
timeperiod 14 & \num{3.320}*** & \\
 & (\num{0.151}) & \\
timeperiod 2 & \num{0.530}*** & \num{0.256}\\
 & (\num{0.070}) & (\num{0.222})\\
timeperiod 3 & \num{1.068}*** & \num{0.807}***\\
 & (\num{0.096}) & (\num{0.225})\\
timeperiod 4 & \num{1.217}*** & \num{1.813}***\\
 & (\num{0.087}) & (\num{0.240})\\
timeperiod 5 & \num{1.558}*** & \\
 & (\num{0.103}) & \\
timeperiod 6 & \num{1.688}*** & \\
 & (\num{0.087}) & \\
timeperiod 7 & \num{1.832}*** & \\
 & (\num{0.068}) & \\
timeperiod 8 & \num{2.041}*** & \\
 & (\num{0.131}) & \\
timeperiod 9 & \num{1.997}*** & \\
 & (\num{0.072}) & \\
Log(scale) & \num{-1.998}*** & \num{-1.072}***\\
 & (\num{0.210}) & (\num{0.114})\\
\midrule
Num.Obs. & \num{2029} & \num{2036}\\
AIC & \num{597.1} & \num{725.3}\\
Log.Lik. & \num{-273.545} & \num{-346.670}\\
\bottomrule
\multicolumn{3}{l}{\rule{0pt}{1em}* p $<$ 0.1, ** p $<$ 0.05, *** p $<$ 0.01}\\
\end{tabular}
\end{table}

\clearpage

Now we estimate the same model for females.

\begin{table}[!h]

\caption{\label{tab:unnamed-chunk-22}PWC (Weibull) - Females only}
\centering
\fontsize{7}{9}\selectfont
\begin{tabular}[t]{lcc}
\toprule
\multicolumn{1}{c}{ } & \multicolumn{1}{c}{PWC 14 steps} & \multicolumn{1}{c}{PWC 4 steps} \\
\cmidrule(l{3pt}r{3pt}){2-2} \cmidrule(l{3pt}r{3pt}){3-3}
  & Model 1 & Model 2\\
\midrule
(Intercept) & \num{3.084}*** & \num{3.780}***\\
 & (\num{0.220}) & (\num{0.615})\\
marstat & \num{0.019} & \num{0.097}*\\
 & (\num{0.019}) & (\num{0.059})\\
contract & \num{0.150} & \num{0.309}\\
 & (\num{0.099}) & (\num{0.251})\\
merged & \num{0.002} & \num{0.001}\\
 & (\num{0.007}) & (\num{0.019})\\
lowgroup & \num{-0.073}* & \num{-0.172}\\
 & (\num{0.038}) & (\num{0.125})\\
classize & \num{-0.001} & \num{-0.002}\\
 & (\num{0.001}) & (\num{0.004})\\
schsize & \num{0.000} & \num{0.000}\\
 & (\num{0.000}) & (\num{0.000})\\
public & \num{0.045} & \num{0.014}\\
 & (\num{0.032}) & (\num{0.103})\\
protest & \num{0.104}** & \num{0.137}\\
 & (\num{0.045}) & (\num{0.143})\\
avgfem & \num{-0.151} & \num{-0.160}\\
 & (\num{0.103}) & (\num{0.347})\\
avgage & \num{0.004} & \num{0.008}\\
 & (\num{0.004}) & (\num{0.010})\\
avglowgr & \num{0.198}** & \num{0.226}\\
 & (\num{0.086}) & (\num{0.326})\\
timeperiod 10 & \num{2.116}*** & \\
 & (\num{0.057}) & \\
timeperiod 11 & \num{2.302}*** & \\
 & (\num{0.064}) & \\
timeperiod 12 & \num{2.437}*** & \\
 & (\num{0.058}) \vphantom{1} & \\
timeperiod 13 & \num{42.067}*** & \\
 & (\num{0.000}) & \\
timeperiod 14 & \num{3.169}*** & \\
 & (\num{0.119}) & \\
timeperiod 2 & \num{0.591}*** & \num{0.045}\\
 & (\num{0.051}) & (\num{0.168})\\
timeperiod 3 & \num{1.002}*** & \num{0.522}***\\
 & (\num{0.060}) & (\num{0.174})\\
timeperiod 4 & \num{1.297}*** & \num{1.500}***\\
 & (\num{0.054}) & (\num{0.195})\\
timeperiod 5 & \num{1.509}*** & \\
 & (\num{0.053}) \vphantom{1} & \\
timeperiod 6 & \num{1.627}*** & \\
 & (\num{0.058}) & \\
timeperiod 7 & \num{1.749}*** & \\
 & (\num{0.053}) & \\
timeperiod 8 & \num{2.137}*** & \\
 & (\num{0.099}) & \\
timeperiod 9 & \num{2.069}*** & \\
 & (\num{0.070}) & \\
Log(scale) & \num{-1.882}*** & \num{-0.874}***\\
 & (\num{0.141}) & (\num{0.060})\\
\midrule
Num.Obs. & \num{4444} & \num{4456}\\
AIC & \num{1619.9} & \num{1922.2}\\
Log.Lik. & \num{-783.936} & \num{-945.106}\\
\bottomrule
\multicolumn{3}{l}{\rule{0pt}{1em}* p $<$ 0.1, ** p $<$ 0.05, *** p $<$ 0.01}\\
\end{tabular}
\end{table}

\clearpage

\begin{enumerate}
\def\labelenumi{\arabic{enumi}.}
\setcounter{enumi}{3}
\tightlist
\item
  Estimate a Cox model and compare the most elaborate specification with
  the results of the PWC model
\end{enumerate}

\begin{table}[!h]

\caption{\label{tab:unnamed-chunk-23}Cox Model}
\centering
\fontsize{8}{10}\selectfont
\begin{tabular}[t]{lcc}
\toprule
  & Model 1 & Model 2\\
\midrule
marstat & \num{-0.007} & \num{0.002}\\
 & (\num{0.024}) & (\num{0.024})\\
gender & \num{-0.129}*** & \num{-0.147}***\\
 & (\num{0.031}) & (\num{0.034})\\
contract & \num{0.224}*** & \num{0.206}***\\
 & (\num{0.050}) & (\num{0.050})\\
lowgroup & \num{-0.004} & \num{0.002}\\
 & (\num{0.030}) & (\num{0.031})\\
classize & \num{0.000} & \num{0.000}\\
 & (\num{0.002}) & (\num{0.002})\\
schsize & \num{0.000} & \num{0.000}\\
 & (\num{0.000}) & (\num{0.000})\\
public & \num{-0.007} & \num{-0.006}\\
 & (\num{0.029}) & (\num{0.029})\\
protest & \num{0.075} & \num{0.038}\\
 & (\num{0.034}) & (\num{0.035})\\
merged &  & \num{-0.009}\\
 &  & (\num{0.006})\\
avgfem &  & \num{0.076}\\
 &  & (\num{0.099})\\
avgage &  & \num{-0.018}***\\
 &  & (\num{0.004})\\
avglowgr &  & \num{-0.071}\\
 &  & (\num{0.096})\\
\midrule
Num.Obs. & \num{6520} & \num{6520}\\
R2 & \num{0.007} & \num{0.012}\\
AIC & \num{99424.7} & \num{99398.2}\\
Log.Lik. & \num{-49704.363} & \num{-49687.124}\\
\bottomrule
\multicolumn{3}{l}{\rule{0pt}{1em}* p $<$ 0.1, ** p $<$ 0.05, *** p $<$ 0.01}\\
\end{tabular}
\end{table}

Comparing the most elaborate specification of the Cox model with the
most elaborate PWC model, we observe an interesting finding. That is,
the model estimates appear to be very sensitive to our choice of model.
For example, gender appears to be significant at the 1\% level for the
PWC model, with a positive sign, and for the Cox model it has the same
significance level, but has a negative sign. This clearly illustrates
the sensitivity of the parameter estimates to the parametric form
imposed.

\clearpage

\begin{enumerate}
\def\labelenumi{\arabic{enumi}.}
\setcounter{enumi}{4}
\tightlist
\item
  Repeat the procedure of question 2 for a Weibull model with (e.g.,
  gamma) unobserved heterogeneity. Compare the estimates of the
  regression coefficients across the models with and without unobserved
  heterogeneity.
\end{enumerate}

As above, but now with a Piecewise Constant (PWC) specification, where
you have an elaborate specification of the baseline hazard (say, 20
dummies).

\begin{table}[!h]

\caption{\label{tab:unnamed-chunk-25}Gamma Weibull Models}
\centering
\fontsize{9}{11}\selectfont
\begin{tabular}[t]{lccc}
\toprule
  & No time dummies & 4 Time Dummies & 22 Time Dummeis\\
\midrule
mu & \num{4.481} & \num{7.907} & \num{1.541}\\
 &  & (\num{1153.496}) & (\num{0.033})\\
sigma & \num{1.654} & \num{0.732} & \num{0.686}\\
 &  & (\num{0.094}) & (\num{0.007})\\
Q & \num{1.000} & \num{1.000} & \num{1.000}\\
gender & \num{-1.034} & \num{-3.114} & \num{0.042}**\\
 &  & (\num{576.748}) & (\num{0.019})\\
AIC & \num{31020.2} & \num{24329.4} & \num{34240.8}\\
\midrule
Log.Lik. & \num{-15507.109} & \num{-12158.676} & \num{-17095.402}\\
N & \num{4474.000} & \num{4456.000} & \num{6501.000}\\
\bottomrule
\multicolumn{4}{l}{\rule{0pt}{1em}* p $<$ 0.1, ** p $<$ 0.05, *** p $<$ 0.01}\\
\end{tabular}
\end{table}

In the above table, we report the three different Gamma-Weibull models,
with gender as an explanatory variable and no time dummies (model 1), 4
time dummies (model 2), and 22 time dummies (model 3). We observe that
the gender coefficient is significant in the third case, indicating that
an extensive and non-parametric baseline hazard helps us identify the
effect of gender. When contrasting the results with the Weibull model
from question 2, we find that the coefficient on gender is again much
smaller, indicating that the results might have been due to spurious
correlation and subject-specific effects, rather than within-subject
variation.

5.2 Compare the estimates of the Cox model (question 4) with the results
of the PWC model with unobserved heterogeneity.

The results from the Cox model in question four come closer to the
results in this question, when focusing on the coefficient on gender.
Still, if we take these results to be the true results, the Cox model
overestimates the coefficient by a factor of roughly three, which is
substantial. That means that also in case of duration data, a non-panel
model cannot easily reflect the within-subjects effect, as we know from
linear and other simpler models.

\clearpage

\begin{enumerate}
\def\labelenumi{\arabic{enumi}.}
\setcounter{enumi}{5}
\tightlist
\item
  Multiple Spells
\end{enumerate}

6.1 Estimate a standard Cox model (PL) and estimate Stratified Cox
models (SPL). Concerning the latter, estimate SPL models, where the
school is the stratum and estimate one where the teacher is the stratum.
Comment on the teacher SPL approach. Compare the PL and the school SPL
estimates. Can you think of a test to test for the relevance of using
the school SPL (rather than doing the PL)?

First we estimate a model with schoolid as a strata. This gives us the
following output (the package we use can not produce standard tables,
hence we show its output):

\begin{verbatim}
## Call:
## survival::coxph(formula = Surv(splength, rcensor) ~ strata(schoolid) + 
##     marstat + gender + contract + lowgroup + classize + schsize + 
##     public + protest, data = dataset %>% filter(sptype == 2) %>% 
##     mutate(rcensor = if_else(rcensor == 1, 0, 1)), cluster = schoolid)
## 
##   n= 6520, number of events= 6324 
## 
##                coef  exp(coef)   se(coef)  robust se      z Pr(>|z|)    
## marstat   0.0171112  1.0172584  0.0284297  0.0349686  0.489 0.624607    
## gender   -0.1523871  0.8586559  0.0369382  0.0462039 -3.298 0.000973 ***
## contract  0.1928554  1.2127074  0.0582814  0.0550957  3.500 0.000465 ***
## lowgroup -0.0274227  0.9729499  0.0341762  0.0395282 -0.694 0.487838    
## classize -0.0006756  0.9993246  0.0021278  0.0020257 -0.334 0.738734    
## schsize  -0.0002969  0.9997031  0.0005075  0.0003643 -0.815 0.415017    
## public   -0.1253421  0.8821950  0.1397192  0.1740631 -0.720 0.471466    
## protest  -0.2016655  0.8173683  0.1689328  0.1500950 -1.344 0.179082    
## ---
## Signif. codes:  0 '***' 0.001 '**' 0.01 '*' 0.05 '.' 0.1 ' ' 1
## 
##          exp(coef) exp(-coef) lower .95 upper .95
## marstat     1.0173     0.9830    0.9499     1.089
## gender      0.8587     1.1646    0.7843     0.940
## contract    1.2127     0.8246    1.0886     1.351
## lowgroup    0.9729     1.0278    0.9004     1.051
## classize    0.9993     1.0007    0.9954     1.003
## schsize     0.9997     1.0003    0.9990     1.000
## public      0.8822     1.1335    0.6272     1.241
## protest     0.8174     1.2234    0.6091     1.097
## 
## Concordance= 0.516  (se = 0.012 )
## Likelihood ratio test= 37.8  on 8 df,   p=8e-06
## Wald test            = 33.26  on 8 df,   p=6e-05
## Score (logrank) test = 38.38  on 8 df,   p=6e-06,   Robust = 29.2  p=3e-04
## 
##   (Note: the likelihood ratio and score tests assume independence of
##      observations within a cluster, the Wald and robust score tests do not).
\end{verbatim}

\begin{verbatim}
## Call:
## survival::coxph(formula = Surv(splength, rcensor) ~ strata(schoolid) + 
##     marstat + gender + contract + lowgroup + classize + schsize + 
##     public + protest + merged + avgfem + avgage + avglowgr, data = dataset %>% 
##     filter(sptype == 2) %>% mutate(rcensor = if_else(rcensor == 
##     1, 0, 1)), cluster = schoolid)
## 
##   n= 6520, number of events= 6324 
## 
##                coef  exp(coef)   se(coef)  robust se      z Pr(>|z|)    
## marstat   0.0180778  1.0182422  0.0284262  0.0350296  0.516 0.605805    
## gender   -0.1518962  0.8590774  0.0369293  0.0460497 -3.299 0.000972 ***
## contract  0.1923285  1.2120687  0.0583153  0.0549965  3.497 0.000470 ***
## lowgroup -0.0255371  0.9747863  0.0341763  0.0393892 -0.648 0.516774    
## classize -0.0006425  0.9993577  0.0021221  0.0019948 -0.322 0.747373    
## schsize  -0.0004039  0.9995962  0.0005133  0.0003913 -1.032 0.301978    
## public   -0.1228420  0.8844034  0.1397313  0.1737639 -0.707 0.479599    
## protest  -0.2045353  0.8150260  0.1689518  0.1514714 -1.350 0.176912    
## merged   -0.0198450  0.9803506  0.0097281  0.0099312 -1.998 0.045689 *  
## avgfem           NA         NA  0.0000000  0.0000000     NA       NA    
## avgage           NA         NA  0.0000000  0.0000000     NA       NA    
## avglowgr         NA         NA  0.0000000  0.0000000     NA       NA    
## ---
## Signif. codes:  0 '***' 0.001 '**' 0.01 '*' 0.05 '.' 0.1 ' ' 1
## 
##          exp(coef) exp(-coef) lower .95 upper .95
## marstat     1.0182     0.9821    0.9507    1.0906
## gender      0.8591     1.1640    0.7849    0.9402
## contract    1.2121     0.8250    1.0882    1.3500
## lowgroup    0.9748     1.0259    0.9024    1.0530
## classize    0.9994     1.0006    0.9955    1.0033
## schsize     0.9996     1.0004    0.9988    1.0004
## public      0.8844     1.1307    0.6291    1.2432
## protest     0.8150     1.2270    0.6057    1.0967
## merged      0.9804     1.0200    0.9615    0.9996
## avgfem          NA         NA        NA        NA
## avgage          NA         NA        NA        NA
## avglowgr        NA         NA        NA        NA
## 
## Concordance= 0.523  (se = 0.012 )
## Likelihood ratio test= 42  on 9 df,   p=3e-06
## Wald test            = 35.67  on 9 df,   p=5e-05
## Score (logrank) test = 42.53  on 9 df,   p=3e-06,   Robust = 31.07  p=3e-04
## 
##   (Note: the likelihood ratio and score tests assume independence of
##      observations within a cluster, the Wald and robust score tests do not).
\end{verbatim}

The estimated model with teachers as the strata is:

\begin{verbatim}
## Call:
## survival::coxph(formula = Surv(splength, rcensor) ~ strata(teachid) + 
##     marstat + gender + contract + lowgroup + classize + schsize + 
##     public + protest, data = dataset %>% filter(sptype == 2) %>% 
##     mutate(rcensor = if_else(rcensor == 1, 0, 1)), cluster = schoolid)
## 
##   n= 6520, number of events= 6324 
## 
##                coef  exp(coef)   se(coef)  robust se      z Pr(>|z|)    
## marstat   3.810e-03  1.004e+00  2.436e-02  2.887e-02  0.132 0.894994    
## gender   -1.357e-01  8.731e-01  3.251e-02  4.001e-02 -3.392 0.000695 ***
## contract  2.280e-01  1.256e+00  5.307e-02  6.491e-02  3.512 0.000445 ***
## lowgroup -1.182e-02  9.882e-01  3.058e-02  3.810e-02 -0.310 0.756286    
## classize -2.504e-05  1.000e+00  1.781e-03  1.935e-03 -0.013 0.989673    
## schsize  -1.006e-04  9.999e-01  1.270e-04  2.149e-04 -0.468 0.639710    
## public   -2.719e-03  9.973e-01  2.939e-02  4.571e-02 -0.059 0.952576    
## protest   7.510e-02  1.078e+00  3.423e-02  5.156e-02  1.457 0.145182    
## ---
## Signif. codes:  0 '***' 0.001 '**' 0.01 '*' 0.05 '.' 0.1 ' ' 1
## 
##          exp(coef) exp(-coef) lower .95 upper .95
## marstat     1.0038     0.9962    0.9486    1.0623
## gender      0.8731     1.1454    0.8072    0.9443
## contract    1.2560     0.7962    1.1060    1.4264
## lowgroup    0.9882     1.0119    0.9171    1.0649
## classize    1.0000     1.0000    0.9962    1.0038
## schsize     0.9999     1.0001    0.9995    1.0003
## public      0.9973     1.0027    0.9118    1.0908
## protest     1.0780     0.9276    0.9744    1.1926
## 
## Concordance= 0.529  (se = 0.007 )
## Likelihood ratio test= 47.01  on 8 df,   p=2e-07
## Wald test            = 30.95  on 8 df,   p=1e-04
## Score (logrank) test = 48.85  on 8 df,   p=7e-08,   Robust = 29.2  p=3e-04
## 
##   (Note: the likelihood ratio and score tests assume independence of
##      observations within a cluster, the Wald and robust score tests do not).
\end{verbatim}

\begin{verbatim}
## Call:
## survival::coxph(formula = Surv(splength, rcensor) ~ strata(teachid) + 
##     marstat + gender + contract + lowgroup + classize + schsize + 
##     public + protest + merged + avgfem + avgage + avglowgr, data = dataset %>% 
##     filter(sptype == 2) %>% mutate(rcensor = if_else(rcensor == 
##     1, 0, 1)), cluster = schoolid)
## 
##   n= 6520, number of events= 6324 
## 
##                coef  exp(coef)   se(coef)  robust se      z Pr(>|z|)    
## marstat   0.0132498  1.0133379  0.0246037  0.0292414  0.453 0.650465    
## gender   -0.1517235  0.8592258  0.0348124  0.0423751 -3.580 0.000343 ***
## contract  0.2114603  1.2354809  0.0533250  0.0637240  3.318 0.000905 ***
## lowgroup -0.0057975  0.9942192  0.0321718  0.0380892 -0.152 0.879022    
## classize -0.0002979  0.9997021  0.0017902  0.0019545 -0.152 0.878852    
## schsize  -0.0002714  0.9997286  0.0001325  0.0001915 -1.417 0.156416    
## public   -0.0006410  0.9993592  0.0296442  0.0439327 -0.015 0.988360    
## protest   0.0388644  1.0396295  0.0351635  0.0526585  0.738 0.460486    
## merged   -0.0105901  0.9894657  0.0056826  0.0073454 -1.442 0.149374    
## avgfem    0.0553857  1.0569482  0.1004482  0.1392552  0.398 0.690831    
## avgage   -0.0186106  0.9815615  0.0035485  0.0049979 -3.724 0.000196 ***
## avglowgr -0.0726250  0.9299495  0.0976814  0.1429178 -0.508 0.611342    
## ---
## Signif. codes:  0 '***' 0.001 '**' 0.01 '*' 0.05 '.' 0.1 ' ' 1
## 
##          exp(coef) exp(-coef) lower .95 upper .95
## marstat     1.0133     0.9868    0.9569    1.0731
## gender      0.8592     1.1638    0.7907    0.9336
## contract    1.2355     0.8094    1.0904    1.3998
## lowgroup    0.9942     1.0058    0.9227    1.0713
## classize    0.9997     1.0003    0.9959    1.0035
## schsize     0.9997     1.0003    0.9994    1.0001
## public      0.9994     1.0006    0.9169    1.0892
## protest     1.0396     0.9619    0.9377    1.1527
## merged      0.9895     1.0106    0.9753    1.0038
## avgfem      1.0569     0.9461    0.8045    1.3886
## avgage      0.9816     1.0188    0.9720    0.9912
## avglowgr    0.9299     1.0753    0.7028    1.2306
## 
## Concordance= 0.541  (se = 0.007 )
## Likelihood ratio test= 81.48  on 12 df,   p=2e-12
## Wald test            = 45.53  on 12 df,   p=8e-06
## Score (logrank) test = 83.08  on 12 df,   p=1e-12,   Robust = 38.96  p=1e-04
## 
##   (Note: the likelihood ratio and score tests assume independence of
##      observations within a cluster, the Wald and robust score tests do not).
\end{verbatim}

Let us first compare the PL model with the SPL model with schoolid as
the strata. The most striking feauture of the PL model was that gender
and contract were highly statistically and economically significant.
This feauture is also observable in the SPL model, and the signs are the
same. Apart from these findings we do not observe any other interesting
findings.

Second, we can compare the teacher stratified model with the school
stratified model. Again we obtain very comparable estimates. This
indicates that choosing a strata does not seem to have an added value.
\clearpage

6.2 Estimate a model with school specific dummies and compare these
estimates with those obtained from the school SPL.

\begin{table}[!h]

\caption{\label{tab:unnamed-chunk-28}Cox Model with School Dummies}
\centering
\fontsize{8}{10}\selectfont
\begin{tabular}[t]{lcc}
\toprule
  & Model 1 & Model 2\\
\midrule
schooldummies & \num{0.001}*** & \num{0.001}***\\
 & (\num{0.000}) & (\num{0.000})\\
marstat & \num{-0.092} & \num{-0.079}\\
 & (\num{0.129}) & (\num{0.129})\\
gender & \num{-0.316}* & \num{-0.392}**\\
 & (\num{0.182}) & (\num{0.192})\\
contract & \num{-0.419} & \num{-0.445}\\
 & (\num{0.706}) & (\num{0.711})\\
lowgroup & \num{0.094} & \num{0.110}\\
 & (\num{0.172}) & (\num{0.179})\\
classize & \num{0.003} & \num{0.004}\\
 & (\num{0.008}) & (\num{0.008})\\
schsize & \num{0.000} & \num{-0.001}\\
 & (\num{0.001}) & (\num{0.001})\\
public & \num{0.027} & \num{0.006}\\
 & (\num{0.166}) & (\num{0.169})\\
protest & \num{-0.002} & \num{-0.036}\\
 & (\num{0.196}) & (\num{0.202})\\
merged &  & \num{-0.008}\\
 &  & (\num{0.036})\\
avgfem &  & \num{0.559}\\
 &  & (\num{0.571})\\
avgage &  & \num{-0.011}\\
 &  & (\num{0.020})\\
avglowgr &  & \num{0.038}\\
 &  & (\num{0.567})\\
\midrule
Num.Obs. & \num{6520} & \num{6520}\\
R2 & \num{0.002} & \num{0.002}\\
AIC & \num{2051.4} & \num{2057.4}\\
Log.Lik. & \num{-1016.708} & \num{-1015.698}\\
\bottomrule
\multicolumn{3}{l}{\rule{0pt}{1em}* p $<$ 0.1, ** p $<$ 0.05, *** p $<$ 0.01}\\
\end{tabular}
\end{table}

The results of the model with school specific dummies compared to the
school stratas show us that the schooldummies are highly significant and
including them seem to have an effect on the gender and contract
variables. Gender becomes less significant (depending on the model 10\%
or 5\% level), contract becomes insignificant and changes sign. Allowing
for schooldummies implies that the base hazard rate per school can be
different, but we do not study the variation of the covariates within
schools.

\clearpage

6.3 Observed sickness patterns vary between schools. This may be due to
sorting effects (bad teachers are the reason why the school scores bad
in absenteeism) and/or the school effects (it is elements of the school
that make some schools worse than others. Can you think of a
test/procedure to shed some more light on this issue?

We decide to test the model specification of schoolid as a strata
against the standard cox model based on the log partial likelihood. This
test indicates whether stratas based on schoolid matter. We implement
this test using the survival package in R. Our test results are shown
below. The results indicate that the standard model is worse compared to
the stratified model (likelihoods differ highly significantly). The
associated p-value is a lot smaller than 0.01, meaning it is a highly
significant result according to the test. However, given the previous
comparison of the model's estimates, we cannot conclude that there might
be sorting effect, even though it intuitively makes sense.

\begin{verbatim}
## Analysis of Deviance Table
##  Cox model: response is  Surv(splength, rcensor)
##  Model 1: ~ marstat + gender + contract + lowgroup + classize + schsize + public + protest + merged + avgfem + avgage + avglowgr
##  Model 2: ~ strata(schoolid) + marstat + gender + contract + lowgroup + classize + schsize + public + protest + merged + avgfem + avgage + avglowgr
##   loglik Chisq Df P(>|Chi|)    
## 1 -49687                       
## 2 -14554 70267  3 < 2.2e-16 ***
## ---
## Signif. codes:  0 '***' 0.001 '**' 0.01 '*' 0.05 '.' 0.1 ' ' 1
\end{verbatim}

\end{document}

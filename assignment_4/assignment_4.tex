% Options for packages loaded elsewhere
\PassOptionsToPackage{unicode}{hyperref}
\PassOptionsToPackage{hyphens}{url}
%
\documentclass[
]{article}
\usepackage{amsmath,amssymb}
\usepackage{lmodern}
\usepackage{iftex}
\ifPDFTeX
  \usepackage[T1]{fontenc}
  \usepackage[utf8]{inputenc}
  \usepackage{textcomp} % provide euro and other symbols
\else % if luatex or xetex
  \usepackage{unicode-math}
  \defaultfontfeatures{Scale=MatchLowercase}
  \defaultfontfeatures[\rmfamily]{Ligatures=TeX,Scale=1}
\fi
% Use upquote if available, for straight quotes in verbatim environments
\IfFileExists{upquote.sty}{\usepackage{upquote}}{}
\IfFileExists{microtype.sty}{% use microtype if available
  \usepackage[]{microtype}
  \UseMicrotypeSet[protrusion]{basicmath} % disable protrusion for tt fonts
}{}
\makeatletter
\@ifundefined{KOMAClassName}{% if non-KOMA class
  \IfFileExists{parskip.sty}{%
    \usepackage{parskip}
  }{% else
    \setlength{\parindent}{0pt}
    \setlength{\parskip}{6pt plus 2pt minus 1pt}}
}{% if KOMA class
  \KOMAoptions{parskip=half}}
\makeatother
\usepackage{xcolor}
\IfFileExists{xurl.sty}{\usepackage{xurl}}{} % add URL line breaks if available
\IfFileExists{bookmark.sty}{\usepackage{bookmark}}{\usepackage{hyperref}}
\hypersetup{
  pdftitle={Applied Microeconometrics - Assignment 4},
  pdfauthor={Walter Verwer (589962) \& Bas Machielsen (590049)},
  hidelinks,
  pdfcreator={LaTeX via pandoc}}
\urlstyle{same} % disable monospaced font for URLs
\usepackage[margin=1in]{geometry}
\usepackage{graphicx}
\makeatletter
\def\maxwidth{\ifdim\Gin@nat@width>\linewidth\linewidth\else\Gin@nat@width\fi}
\def\maxheight{\ifdim\Gin@nat@height>\textheight\textheight\else\Gin@nat@height\fi}
\makeatother
% Scale images if necessary, so that they will not overflow the page
% margins by default, and it is still possible to overwrite the defaults
% using explicit options in \includegraphics[width, height, ...]{}
\setkeys{Gin}{width=\maxwidth,height=\maxheight,keepaspectratio}
% Set default figure placement to htbp
\makeatletter
\def\fps@figure{htbp}
\makeatother
\setlength{\emergencystretch}{3em} % prevent overfull lines
\providecommand{\tightlist}{%
  \setlength{\itemsep}{0pt}\setlength{\parskip}{0pt}}
\setcounter{secnumdepth}{-\maxdimen} % remove section numbering
\usepackage{booktabs}
\usepackage{longtable}
\usepackage{array}
\usepackage{multirow}
\usepackage{wrapfig}
\usepackage{float}
\usepackage{colortbl}
\usepackage{pdflscape}
\usepackage{tabu}
\usepackage{threeparttable}
\usepackage{threeparttablex}
\usepackage[normalem]{ulem}
\usepackage{makecell}
\usepackage{xcolor}
\usepackage{siunitx}
\newcolumntype{d}{S[input-symbols = ()]}
\ifLuaTeX
  \usepackage{selnolig}  % disable illegal ligatures
\fi

\title{Applied Microeconometrics - Assignment 4}
\author{Walter Verwer (589962) \& Bas Machielsen (590049)}
\date{\today}

\begin{document}
\maketitle

\begin{enumerate}
\def\labelenumi{\arabic{enumi}.}
\tightlist
\item
  Describe the sickness spell data, i.e.~do a simple listing of the
  survivor function and plot the hazard rate and the survivor function.
  Make separate plots for the first two weeks and for the first year.
  Also plot the hazard by different subgroups (for instance gender) and
  test whether the survival curves are the same for the different
  subgroups.
\end{enumerate}

First, we plot the exit rates:

\includegraphics[width=200px,height=150px]{assignment_4_files/figure-latex/unnamed-chunk-2-1}
\includegraphics[width=200px,height=150px]{assignment_4_files/figure-latex/unnamed-chunk-2-2}

\includegraphics[width=200px,height=150px]{assignment_4_files/figure-latex/unnamed-chunk-3-1}
\includegraphics[width=200px,height=150px]{assignment_4_files/figure-latex/unnamed-chunk-3-2}

Then, we plot the survivor functions:

\includegraphics[width=200px,height=150px]{assignment_4_files/figure-latex/unnamed-chunk-4-1}
\includegraphics[width=200px,height=150px]{assignment_4_files/figure-latex/unnamed-chunk-4-2}

\includegraphics[width=200px,height=150px]{assignment_4_files/figure-latex/unnamed-chunk-5-1}
\includegraphics[width=200px,height=150px]{assignment_4_files/figure-latex/unnamed-chunk-5-2}

\clearpage

2.1 Estimate a Weibull and an Exponential model for sickness spells.
Start with a very simple specification and you only include one
regressor and subsequently add more regressors. Comment on the change in
the Weibull parameters and the regression parameters when you add more
variables to the model. Compare the estimates of both models.

\begin{table}[!h]

\caption{\label{tab:unnamed-chunk-7}Weibull Models}
\centering
\resizebox{\linewidth}{!}{
\begin{tabular}[t]{lcccccc}
\toprule
  & Model 1 & Model 2 & Model 3 & Model 4 & Model 5 & Model 6\\
\midrule
(Intercept) & \num{1.938}*** & \num{1.795}*** & \num{2.280}*** & \num{2.279}*** & \num{2.334}*** & \num{0.979}*\\
 & (\num{0.127}) & (\num{0.195}) & (\num{0.213}) & (\num{0.213}) & (\num{0.251}) & (\num{0.530})\\
gender & \num{0.243}*** & \num{0.254}*** & \num{0.273}*** & \num{0.274}*** & \num{0.278}*** & \num{0.319}***\\
 & (\num{0.077}) & (\num{0.078}) & (\num{0.078}) & (\num{0.087}) & (\num{0.087}) & (\num{0.092})\\
Log(scale) & \num{0.490}*** & \num{0.490}*** & \num{0.487}*** & \num{0.487}*** & \num{0.486}*** & \num{0.483}***\\
 & (\num{0.013}) & (\num{0.013}) & (\num{0.013}) & (\num{0.013}) & (\num{0.013}) & (\num{0.013})\\
marstat &  & \num{0.069} & \num{0.055} & \num{0.055} & \num{0.049} & \num{0.031}\\
 &  & (\num{0.066}) & (\num{0.066}) & (\num{0.066}) & (\num{0.066}) & (\num{0.068})\\
contract &  &  & \num{-0.467}*** & \num{-0.467}*** & \num{-0.467}*** & \num{-0.423}***\\
 &  &  & (\num{0.093}) & (\num{0.093}) & (\num{0.094}) & (\num{0.093})\\
lowgroup &  &  &  & \num{-0.004} & \num{-0.005} & \num{-0.020}\\
 &  &  &  & (\num{0.079}) & (\num{0.079}) & (\num{0.080})\\
classize &  &  &  &  & \num{0.002} & \num{0.002}\\
 &  &  &  &  & (\num{0.005}) & (\num{0.005})\\
schsize &  &  &  &  & \num{0.000} & \num{0.000}\\
 &  &  &  &  & (\num{0.000}) & (\num{0.000})\\
public &  &  &  &  & \num{-0.056} & \num{-0.048}\\
 &  &  &  &  & (\num{0.091}) & (\num{0.089})\\
protest &  &  &  &  & \num{-0.153} & \num{-0.078}\\
 &  &  &  &  & (\num{0.106}) & (\num{0.108})\\
merged &  &  &  &  &  & \num{0.005}\\
 &  &  &  &  &  & (\num{0.013})\\
avgfem &  &  &  &  &  & \num{-0.206}\\
 &  &  &  &  &  & (\num{0.284})\\
avgage &  &  &  &  &  & \num{0.034}***\\
 &  &  &  &  &  & (\num{0.011})\\
avglowgr &  &  &  &  &  & \num{0.122}\\
 &  &  &  &  &  & (\num{0.259})\\
\midrule
Num.Obs. & \num{6520} & \num{6520} & \num{6520} & \num{6520} & \num{6520} & \num{6520}\\
AIC & \num{44443.9} & \num{44442.6} & \num{44415.1} & \num{44417.1} & \num{44416.8} & \num{44383.2}\\
Log.Lik. & \num{-22218.934} & \num{-22217.295} & \num{-22202.540} & \num{-22202.537} & \num{-22198.402} & \num{-22177.593}\\
\bottomrule
\multicolumn{7}{l}{\rule{0pt}{1em}* p $<$ 0.1, ** p $<$ 0.05, *** p $<$ 0.01}\\
\end{tabular}}
\end{table}

\clearpage

\begin{table}[!h]

\caption{\label{tab:unnamed-chunk-9}Exponential Models}
\centering
\resizebox{\linewidth}{!}{
\begin{tabular}[t]{lcccccc}
\toprule
  & Model 1 & Model 2 & Model 3 & Model 4 & Model 5 & Model 6\\
\midrule
(Intercept) & \num{2.435}*** & \num{2.247}*** & \num{2.852}*** & \num{2.848}*** & \num{2.909}*** & \num{1.601}**\\
 & (\num{0.164}) & (\num{0.236}) & (\num{0.255}) & (\num{0.254}) & (\num{0.290}) & (\num{0.633})\\
gender & \num{0.267}*** & \num{0.280}*** & \num{0.303}*** & \num{0.324}*** & \num{0.332}*** & \num{0.377}***\\
 & (\num{0.097}) & (\num{0.098}) & (\num{0.097}) & (\num{0.104}) & (\num{0.103}) & (\num{0.108})\\
marstat &  & \num{0.091} & \num{0.077} & \num{0.076} & \num{0.077} & \num{0.058}\\
 &  & (\num{0.080}) & (\num{0.079}) & (\num{0.079}) & (\num{0.078}) & (\num{0.080})\\
contract &  &  & \num{-0.594}*** & \num{-0.595}*** & \num{-0.600}*** & \num{-0.537}***\\
 &  &  & (\num{0.112}) & (\num{0.111}) & (\num{0.112}) & (\num{0.116})\\
lowgroup &  &  &  & \num{-0.044} & \num{-0.043} & \num{-0.063}\\
 &  &  &  & (\num{0.096}) & (\num{0.097}) & (\num{0.098})\\
classize &  &  &  &  & \num{0.003} & \num{0.003}\\
 &  &  &  &  & (\num{0.005}) & (\num{0.005})\\
schsize &  &  &  &  & \num{0.000} & \num{0.000}\\
 &  &  &  &  & (\num{0.000}) & (\num{0.000})\\
public &  &  &  &  & \num{-0.116} & \num{-0.093}\\
 &  &  &  &  & (\num{0.103}) & (\num{0.104})\\
protest &  &  &  &  & \num{-0.160} & \num{-0.083}\\
 &  &  &  &  & (\num{0.124}) & (\num{0.128})\\
merged &  &  &  &  &  & \num{-0.004}\\
 &  &  &  &  &  & (\num{0.015})\\
avgfem &  &  &  &  &  & \num{-0.237}\\
 &  &  &  &  &  & (\num{0.335})\\
avgage &  &  &  &  &  & \num{0.033}***\\
 &  &  &  &  &  & (\num{0.012})\\
avglowgr &  &  &  &  &  & \num{0.091}\\
 &  &  &  &  &  & (\num{0.297})\\
\midrule
Num.Obs. & \num{6520} & \num{6520} & \num{6520} & \num{6520} & \num{6520} & \num{6520}\\
AIC & \num{49147.6} & \num{49133.1} & \num{49022.5} & \num{49022.1} & \num{48996.4} & \num{48893.8}\\
Log.Lik. & \num{-24571.812} & \num{-24563.554} & \num{-24507.235} & \num{-24506.073} & \num{-24489.220} & \num{-24433.900}\\
\bottomrule
\multicolumn{7}{l}{\rule{0pt}{1em}* p $<$ 0.1, ** p $<$ 0.05, *** p $<$ 0.01}\\
\end{tabular}}
\end{table}

\clearpage

2.2 Estimate separate Weibull models for males and females. Comment on
the results (is it better to estimate separate models for males and
females?) Estimate the Weibull duration model for other subgroups that
may differ in their behavior and where the baseline hazard may differ.

First, we estimate a model for males:

\begin{table}[!h]

\caption{\label{tab:unnamed-chunk-11}Weibull Models - Males Only}
\centering
\resizebox{\linewidth}{!}{
\begin{tabular}[t]{lcccccc}
\toprule
  & Model 1 & Model 2 & Model 3 & Model 4 & Model 5 & Model 6\\
\midrule
(Intercept) & \num{2.204}*** & \num{2.446}*** & \num{3.162}*** & \num{3.149}*** & \num{3.363}*** & \num{1.558}*\\
 & (\num{0.048}) & (\num{0.329}) & (\num{0.394}) & (\num{0.394}) & (\num{0.419}) & (\num{0.904})\\
Log(scale) & \num{0.461}*** & \num{0.460}*** & \num{0.458}*** & \num{0.457}*** & \num{0.456}*** & \num{0.449}***\\
 & (\num{0.024}) & (\num{0.024}) & (\num{0.024}) & (\num{0.024}) & (\num{0.024}) & (\num{0.024})\\
marstat &  & \num{-0.125} & \num{-0.138} & \num{-0.138} & \num{-0.150} & \num{-0.149}\\
 &  & (\num{0.168}) & (\num{0.165}) & (\num{0.166}) & (\num{0.161}) & (\num{0.157})\\
contract &  &  & \num{-0.675}*** & \num{-0.685}*** & \num{-0.631}*** & \num{-0.559}***\\
 &  &  & (\num{0.169}) & (\num{0.174}) & (\num{0.179}) & (\num{0.188})\\
lowgroup &  &  &  & \num{0.083} & \num{0.084} & \num{0.094}\\
 &  &  &  & (\num{0.122}) & (\num{0.122}) & (\num{0.123})\\
classize &  &  &  &  & \num{-0.009} & \num{-0.008}\\
 &  &  &  &  & (\num{0.007}) & (\num{0.007})\\
schsize &  &  &  &  & \num{0.000} & \num{0.000}\\
 &  &  &  &  & (\num{0.000}) & (\num{0.000})\\
public &  &  &  &  & \num{-0.073} & \num{-0.036}\\
 &  &  &  &  & (\num{0.128}) & (\num{0.132})\\
protest &  &  &  &  & \num{0.009} & \num{0.032}\\
 &  &  &  &  & (\num{0.180}) & (\num{0.174})\\
merged &  &  &  &  &  & \num{0.035}\\
 &  &  &  &  &  & (\num{0.028})\\
avgfem &  &  &  &  &  & \num{0.457}\\
 &  &  &  &  &  & (\num{0.400})\\
avgage &  &  &  &  &  & \num{0.041}**\\
 &  &  &  &  &  & (\num{0.016})\\
avglowgr &  &  &  &  &  & \num{-0.357}\\
 &  &  &  &  &  & (\num{0.410})\\
\midrule
Num.Obs. & \num{2046} & \num{2046} & \num{2046} & \num{2046} & \num{2046} & \num{2046}\\
AIC & \num{13422.3} & \num{13422.4} & \num{13416.2} & \num{13417.1} & \num{13421.4} & \num{13405.4}\\
Log.Lik. & \num{-6709.152} & \num{-6708.217} & \num{-6704.101} & \num{-6703.534} & \num{-6701.689} & \num{-6689.697}\\
\bottomrule
\multicolumn{7}{l}{\rule{0pt}{1em}* p $<$ 0.1, ** p $<$ 0.05, *** p $<$ 0.01}\\
\end{tabular}}
\end{table}

\clearpage

Next, we estimate a model for females:

\begin{table}[!h]

\caption{\label{tab:unnamed-chunk-13}Weibull Models - Females Only}
\centering
\resizebox{\linewidth}{!}{
\begin{tabular}[t]{lcccccc}
\toprule
  & Model 1 & Model 2 & Model 3 & Model 4 & Model 5 & Model 6\\
\midrule
(Intercept) & \num{2.414}*** & \num{2.228}*** & \num{2.715}*** & \num{2.755}*** & \num{2.763}*** & \num{1.729}***\\
 & (\num{0.042}) & (\num{0.138}) & (\num{0.174}) & (\num{0.197}) & (\num{0.260}) & (\num{0.643})\\
Log(scale) & \num{0.503}*** & \num{0.502}*** & \num{0.499}*** & \num{0.499}*** & \num{0.498}*** & \num{0.494}***\\
 & (\num{0.016}) & (\num{0.016}) & (\num{0.016}) & (\num{0.016}) & (\num{0.016}) & (\num{0.016})\\
marstat &  & \num{0.106} & \num{0.093} & \num{0.092} & \num{0.076} & \num{0.058}\\
 &  & (\num{0.074}) & (\num{0.074}) & (\num{0.074}) & (\num{0.073}) & (\num{0.076})\\
contract &  &  & \num{-0.435}*** & \num{-0.435}*** & \num{-0.433}*** & \num{-0.393}***\\
 &  &  & (\num{0.103}) & (\num{0.102}) & (\num{0.106}) & (\num{0.103})\\
lowgroup &  &  &  & \num{-0.048} & \num{-0.038} & \num{-0.081}\\
 &  &  &  & (\num{0.099}) & (\num{0.099}) & (\num{0.103})\\
classize &  &  &  &  & \num{0.005} & \num{0.006}\\
 &  &  &  &  & (\num{0.006}) & (\num{0.006})\\
schsize &  &  &  &  & \num{0.000} & \num{0.000}\\
 &  &  &  &  & (\num{0.000}) & (\num{0.000})\\
public &  &  &  &  & \num{-0.039} & \num{-0.030}\\
 &  &  &  &  & (\num{0.116}) & (\num{0.114})\\
protest &  &  &  &  & \num{-0.211}* & \num{-0.123}\\
 &  &  &  &  & (\num{0.127}) & (\num{0.133})\\
merged &  &  &  &  &  & \num{-0.008}\\
 &  &  &  &  &  & (\num{0.014})\\
avgfem &  &  &  &  &  & \num{-0.557}\\
 &  &  &  &  &  & (\num{0.383})\\
avgage &  &  &  &  &  & \num{0.030}**\\
 &  &  &  &  &  & (\num{0.013})\\
avglowgr &  &  &  &  &  & \num{0.403}\\
 &  &  &  &  &  & (\num{0.334})\\
\midrule
Num.Obs. & \num{4474} & \num{4474} & \num{4474} & \num{4474} & \num{4474} & \num{4474}\\
AIC & \num{31018.2} & \num{31013.9} & \num{30994.0} & \num{30995.4} & \num{30991.3} & \num{30969.4}\\
Log.Lik. & \num{-15507.109} & \num{-15503.936} & \num{-15493.017} & \num{-15492.706} & \num{-15486.663} & \num{-15471.686}\\
\bottomrule
\multicolumn{7}{l}{\rule{0pt}{1em}* p $<$ 0.1, ** p $<$ 0.05, *** p $<$ 0.01}\\
\end{tabular}}
\end{table}

Now, given that under question 1 we observed a difference between the
exit rates and the survival functions for the different contract types,
we have estimated Weibull models for the contract types. However, our
results are difficult to compare. The reason for this is that the models
for the temporary and mixed contract have very little observations.
Especially, the mixed contract has too little (36 observations). For
this reason we can not make an educated comparison for this subgroup.

\begin{table}[!h]

\caption{\label{tab:unnamed-chunk-14}Weibull Models - Models for the all contract types}
\centering
\fontsize{8}{10}\selectfont
\begin{tabular}[t]{lccc}
\toprule
\multicolumn{1}{c}{ } & \multicolumn{1}{c}{Fixed contract } & \multicolumn{1}{c}{Temporary contract} & \multicolumn{1}{c}{Mixed contract} \\
\cmidrule(l{3pt}r{3pt}){2-2} \cmidrule(l{3pt}r{3pt}){3-3} \cmidrule(l{3pt}r{3pt}){4-4}
  & Model 1 & Model 2 & Model 3\\
\midrule
(Intercept) & \num{0.526} & \num{0.290} & \num{2.377}\\
 & (\num{0.523}) & (\num{1.742}) & (\num{2.100})\\
marstat & \num{0.019} & \num{0.439}** & \num{-0.323}\\
 & (\num{0.070}) & (\num{0.220}) & (\num{0.225})\\
gender & \num{0.304}*** & \num{0.781}** & \num{0.616}\\
 & (\num{0.095}) & (\num{0.309}) & (\num{0.629})\\
lowgroup & \num{-0.010} & \num{-0.083} & \num{-0.089}\\
 & (\num{0.083}) & (\num{0.271}) & (\num{0.739})\\
classize & \num{0.003} & \num{0.002} & \num{0.008}\\
 & (\num{0.005}) & (\num{0.009}) & (\num{0.035})\\
schsize & \num{0.000} & \num{0.000} & \num{0.005}*\\
 & (\num{0.000}) & (\num{0.001}) & (\num{0.003})\\
public & \num{-0.041} & \num{-0.087} & \num{-0.464}**\\
 & (\num{0.090}) & (\num{0.264}) & (\num{0.189})\\
protest & \num{-0.085} & \num{0.154} & \num{0.367}\\
 & (\num{0.108}) & (\num{0.372}) & (\num{0.522})\\
merged & \num{0.002} & \num{0.056} & \num{-0.013}\\
 & (\num{0.014}) & (\num{0.046}) & (\num{0.052})\\
avgfem & \num{-0.172} & \num{-0.796} & \num{-1.661}*\\
 & (\num{0.285}) & (\num{0.934}) & (\num{1.003})\\
avgage & \num{0.035}*** & \num{-0.013} & \num{-0.045}\\
 & (\num{0.011}) & (\num{0.038}) & (\num{0.040})\\
avglowgr & \num{0.080} & \num{0.978} & \num{0.581}\\
 & (\num{0.262}) & (\num{0.820}) & (\num{1.867})\\
Log(scale) & \num{0.490}*** & \num{0.317}*** & \num{-0.226}*\\
 & (\num{0.014}) & (\num{0.050}) & (\num{0.127})\\
\midrule
Num.Obs. & \num{6196} & \num{288} & \num{36}\\
AIC & \num{42336.4} & \num{1838.7} & \num{202.8}\\
Log.Lik. & \num{-21155.217} & \num{-906.333} & \num{-88.386}\\
\bottomrule
\multicolumn{4}{l}{\rule{0pt}{1em}* p $<$ 0.1, ** p $<$ 0.05, *** p $<$ 0.01}\\
\end{tabular}
\end{table}

\clearpage

3.1 Estimate a Piece Wise Constant (PWC) model for the entire sample.
Use the stsplit command to create multiple record data. You can have as
many steps as the data allow you to take, but first start with only a
few (3 or 4 steps). Next estimate a model with 15-20 steps, or even
more. Plot de duration pattern implied by the estimates and comment on
these and the regression parameters. How do the regression parameters
(\(\beta\)) compare with those of the Weibull model?

First, we estimate a piece wise constant model.

\includegraphics[width=200px,height=150px]{assignment_4_files/figure-latex/unnamed-chunk-15-1}
\includegraphics[width=200px,height=150px]{assignment_4_files/figure-latex/unnamed-chunk-15-2}

Above, we plot the hazard rate and survival function of our estimated
piecewise constant model. The piecewise-constant model coefficient
estimates can be found in the table below. As the attrition from the
sample is very high in the starting period, our Weibull model predicts a
high hazard rate, and thus high attrition in the first period.
Afterwards, piecewise, it attempts to catch up the imbalances by fitting
a new Weibull distribution and hazard rate (implied by the new constant
term), which significantly increases the hazard rate relative to the
previous period. This proves the model does catch up with the decay in
the data that is not as fastly decaying as implied by a Weibull model
under the previous parameters.

One other consequence of including a dummy for the first twenty days is
an underestimation of the survival probability in the early period. Even
though it mildly realistic, it underestimates survival by only focusing
on the first 20 days, under which attrition is very high. We repeat this
exercise again with a piecewise constant model with more dummies, but
omit the plot for brevity's sake.

\begin{table}[!h]

\caption{\label{tab:unnamed-chunk-17}PWC (Weibull)}
\centering
\fontsize{8}{10}\selectfont
\begin{tabular}[t]{lcc}
\toprule
\multicolumn{1}{c}{ } & \multicolumn{1}{c}{PWC 14 steps} & \multicolumn{1}{c}{PWC 4 steps} \\
\cmidrule(l{3pt}r{3pt}){2-2} \cmidrule(l{3pt}r{3pt}){3-3}
  & Model 1 & Model 2\\
\midrule
(Intercept) & \num{1.109}*** & \num{1.024}***\\
 & (\num{0.199}) & (\num{0.181})\\
gender & \num{0.056}* & \num{0.054}*\\
 & (\num{0.033}) & (\num{0.032})\\
marstat & \num{-0.032} & \num{-0.024}\\
 & (\num{0.024}) & (\num{0.023})\\
contract & \num{-0.149}*** & \num{-0.114}**\\
 & (\num{0.044}) & (\num{0.045})\\
merged & \num{0.007} & \num{0.012}*\\
 & (\num{0.007}) & (\num{0.007})\\
lowgroup & \num{0.012} & \num{0.005}\\
 & (\num{0.032}) & (\num{0.031})\\
classize & \num{-0.001} & \num{0.000}\\
 & (\num{0.001}) & (\num{0.001})\\
schsize & \num{0.000} & \num{0.000}\\
 & (\num{0.000}) & (\num{0.000})\\
public & \num{0.037} & \num{0.035}\\
 & (\num{0.038}) & (\num{0.034})\\
protest & \num{-0.005} & \num{0.001}\\
 & (\num{0.041}) & (\num{0.039})\\
avgfem & \num{0.066} & \num{0.014}\\
 & (\num{0.113}) & (\num{0.104})\\
avgage & \num{0.015}*** & \num{0.014}***\\
 & (\num{0.004}) & (\num{0.004})\\
avglowgr & \num{-0.010} & \num{0.030}\\
 & (\num{0.122}) & (\num{0.115})\\
timeperiod 10 & \num{4.632}*** & \\
 & (\num{0.246}) & \\
timeperiod 11 & \num{4.397}*** & \\
 & (\num{0.193}) & \\
timeperiod 12 & \num{4.938}*** & \\
 & (\num{0.250}) & \\
timeperiod 13 & \num{4.137}*** & \\
 & (\num{0.038}) & \\
timeperiod 14 & \num{5.551}*** & \\
 & (\num{0.305}) & \\
timeperiod 2 & \num{2.058}*** & \num{1.955}***\\
 & (\num{0.032}) & (\num{0.029})\\
timeperiod 3 & \num{2.704}*** & \num{2.967}***\\
 & (\num{0.052}) & (\num{0.048})\\
timeperiod 4 & \num{3.150}*** & \num{4.334}***\\
 & (\num{0.075}) & (\num{0.075})\\
timeperiod 5 & \num{3.412}*** & \\
 & (\num{0.080}) & \\
timeperiod 6 & \num{3.731}*** & \\
 & (\num{0.121}) & \\
timeperiod 7 & \num{4.158}*** & \\
 & (\num{0.157}) & \\
timeperiod 8 & \num{3.736}*** & \\
 & (\num{0.089}) & \\
timeperiod 9 & \num{4.550}*** & \\
 & (\num{0.268}) & \\
\midrule
Num.Obs. & \num{6473} & \num{6520}\\
AIC & \num{36079.3} & \num{36146.3}\\
Log.Lik. & \num{-18013.641} & \num{-18057.168}\\
\bottomrule
\multicolumn{3}{l}{\rule{0pt}{1em}* p $<$ 0.1, ** p $<$ 0.05, *** p $<$ 0.01}\\
\end{tabular}
\end{table}

In the above table, we show the estimates of two piece-wise constant
models. For the first model, we have plotted the hazard rate and
survival function, for the second, we omitted it for brevitiy's sake. We
see that the hazard rate is increasing over time, in general, though not
monotonically. The estimated survival curve is just the integral over
all these hazard rates, and show a very fast tendency to go to zero. As
before, this makes sense, as this is the case in the data, but piecewise
dummies focusing on a small first period make the fitted distribution
such that the survival rate is approaching zero in a very short period.

\clearpage

3.2 Estimate separate models for males and females.

First we estimate a PWC model for males.

\begin{table}[!h]

\caption{\label{tab:unnamed-chunk-19}PWC (Weibull) - Males only}
\centering
\fontsize{8}{10}\selectfont
\begin{tabular}[t]{lcc}
\toprule
\multicolumn{1}{c}{ } & \multicolumn{1}{c}{PWC 14 steps} & \multicolumn{1}{c}{PWC 4 steps} \\
\cmidrule(l{3pt}r{3pt}){2-2} \cmidrule(l{3pt}r{3pt}){3-3}
  & Model 1 & Model 2\\
\midrule
(Intercept) & \num{-850.507}*** & \num{-1.016}\\
 & (\num{0.322}) & (\num{0.958})\\
marstat & \num{-0.017} & \num{0.260}**\\
 & (\num{0.060}) & (\num{0.130})\\
contract & \num{853.775}*** & \num{5.494}***\\
 & (\num{0.000}) & (\num{0.661})\\
merged & \num{0.006} & \num{0.017}\\
 & (\num{0.014}) & (\num{0.022})\\
lowgroup & \num{0.067} & \num{-0.082}\\
 & (\num{0.045}) & (\num{0.097})\\
classize & \num{0.001} & \num{-0.008}\\
 & (\num{0.003}) & (\num{0.007})\\
schsize & \num{0.000} & \num{0.000}\\
 & (\num{0.000}) & (\num{0.000})\\
public & \num{0.075} & \num{-0.103}\\
 & (\num{0.054}) & (\num{0.165})\\
protest & \num{-0.049} & \num{-0.341}**\\
 & (\num{0.052}) & (\num{0.145})\\
avgfem & \num{-0.009} & \num{-0.445}\\
 & (\num{0.190}) & (\num{0.457})\\
avgage & \num{0.003} & \num{-0.012}\\
 & (\num{0.006}) & (\num{0.015})\\
avglowgr & \num{-0.086} & \num{0.172}\\
 & (\num{0.180}) & (\num{0.338})\\
timeperiod 10 & \num{2.133}*** & \\
 & (\num{0.102}) & \\
timeperiod 11 & \num{2.191}*** & \\
 & (\num{0.105}) & \\
timeperiod 12 & \num{2.437}*** & \\
 & (\num{0.090}) & \\
timeperiod 14 & \num{3.320}*** & \\
 & (\num{0.151}) & \\
timeperiod 2 & \num{0.530}*** & \num{0.256}\\
 & (\num{0.070}) & (\num{0.222})\\
timeperiod 3 & \num{1.068}*** & \num{0.807}***\\
 & (\num{0.096}) & (\num{0.225})\\
timeperiod 4 & \num{1.217}*** & \num{1.813}***\\
 & (\num{0.087}) & (\num{0.240})\\
timeperiod 5 & \num{1.558}*** & \\
 & (\num{0.103}) & \\
timeperiod 6 & \num{1.688}*** & \\
 & (\num{0.087}) & \\
timeperiod 7 & \num{1.832}*** & \\
 & (\num{0.068}) & \\
timeperiod 8 & \num{2.041}*** & \\
 & (\num{0.131}) & \\
timeperiod 9 & \num{1.997}*** & \\
 & (\num{0.072}) & \\
Log(scale) & \num{-1.998}*** & \num{-1.072}***\\
 & (\num{0.210}) & (\num{0.114})\\
\midrule
Num.Obs. & \num{2029} & \num{2036}\\
AIC & \num{597.1} & \num{725.3}\\
Log.Lik. & \num{-273.545} & \num{-346.670}\\
\bottomrule
\multicolumn{3}{l}{\rule{0pt}{1em}* p $<$ 0.1, ** p $<$ 0.05, *** p $<$ 0.01}\\
\end{tabular}
\end{table}

\clearpage

Now we estimate the same model for females.

\begin{table}[!h]

\caption{\label{tab:unnamed-chunk-21}PWC (Weibull) - Females only}
\centering
\fontsize{7}{9}\selectfont
\begin{tabular}[t]{lcc}
\toprule
\multicolumn{1}{c}{ } & \multicolumn{1}{c}{PWC 14 steps} & \multicolumn{1}{c}{PWC 4 steps} \\
\cmidrule(l{3pt}r{3pt}){2-2} \cmidrule(l{3pt}r{3pt}){3-3}
  & Model 1 & Model 2\\
\midrule
(Intercept) & \num{3.084}*** & \num{3.780}***\\
 & (\num{0.220}) & (\num{0.615})\\
marstat & \num{0.019} & \num{0.097}*\\
 & (\num{0.019}) & (\num{0.059})\\
contract & \num{0.150} & \num{0.309}\\
 & (\num{0.099}) & (\num{0.251})\\
merged & \num{0.002} & \num{0.001}\\
 & (\num{0.007}) & (\num{0.019})\\
lowgroup & \num{-0.073}* & \num{-0.172}\\
 & (\num{0.038}) & (\num{0.125})\\
classize & \num{-0.001} & \num{-0.002}\\
 & (\num{0.001}) & (\num{0.004})\\
schsize & \num{0.000} & \num{0.000}\\
 & (\num{0.000}) & (\num{0.000})\\
public & \num{0.045} & \num{0.014}\\
 & (\num{0.032}) & (\num{0.103})\\
protest & \num{0.104}** & \num{0.137}\\
 & (\num{0.045}) & (\num{0.143})\\
avgfem & \num{-0.151} & \num{-0.160}\\
 & (\num{0.103}) & (\num{0.347})\\
avgage & \num{0.004} & \num{0.008}\\
 & (\num{0.004}) & (\num{0.010})\\
avglowgr & \num{0.198}** & \num{0.226}\\
 & (\num{0.086}) & (\num{0.326})\\
timeperiod 10 & \num{2.116}*** & \\
 & (\num{0.057}) & \\
timeperiod 11 & \num{2.302}*** & \\
 & (\num{0.064}) & \\
timeperiod 12 & \num{2.437}*** & \\
 & (\num{0.058}) \vphantom{1} & \\
timeperiod 13 & \num{42.067}*** & \\
 & (\num{0.000}) & \\
timeperiod 14 & \num{3.169}*** & \\
 & (\num{0.119}) & \\
timeperiod 2 & \num{0.591}*** & \num{0.045}\\
 & (\num{0.051}) & (\num{0.168})\\
timeperiod 3 & \num{1.002}*** & \num{0.522}***\\
 & (\num{0.060}) & (\num{0.174})\\
timeperiod 4 & \num{1.297}*** & \num{1.500}***\\
 & (\num{0.054}) & (\num{0.195})\\
timeperiod 5 & \num{1.509}*** & \\
 & (\num{0.053}) \vphantom{1} & \\
timeperiod 6 & \num{1.627}*** & \\
 & (\num{0.058}) & \\
timeperiod 7 & \num{1.749}*** & \\
 & (\num{0.053}) & \\
timeperiod 8 & \num{2.137}*** & \\
 & (\num{0.099}) & \\
timeperiod 9 & \num{2.069}*** & \\
 & (\num{0.070}) & \\
Log(scale) & \num{-1.882}*** & \num{-0.874}***\\
 & (\num{0.141}) & (\num{0.060})\\
\midrule
Num.Obs. & \num{4444} & \num{4456}\\
AIC & \num{1619.9} & \num{1922.2}\\
Log.Lik. & \num{-783.936} & \num{-945.106}\\
\bottomrule
\multicolumn{3}{l}{\rule{0pt}{1em}* p $<$ 0.1, ** p $<$ 0.05, *** p $<$ 0.01}\\
\end{tabular}
\end{table}

\clearpage

\begin{enumerate}
\def\labelenumi{\arabic{enumi}.}
\setcounter{enumi}{3}
\tightlist
\item
  Estimate a Cox model and compare the most elaborate specification with
  the results of the PWC model
\end{enumerate}

\begin{table}[!h]

\caption{\label{tab:unnamed-chunk-22}Cox Model}
\centering
\fontsize{8}{10}\selectfont
\begin{tabular}[t]{lcc}
\toprule
  & Model 1 & Model 2\\
\midrule
marstat & \num{-0.007} & \num{0.002}\\
 & (\num{0.024}) & (\num{0.024})\\
gender & \num{-0.129}*** & \num{-0.147}***\\
 & (\num{0.031}) & (\num{0.034})\\
contract & \num{0.224}*** & \num{0.206}***\\
 & (\num{0.050}) & (\num{0.050})\\
lowgroup & \num{-0.004} & \num{0.002}\\
 & (\num{0.030}) & (\num{0.031})\\
classize & \num{0.000} & \num{0.000}\\
 & (\num{0.002}) & (\num{0.002})\\
schsize & \num{0.000} & \num{0.000}\\
 & (\num{0.000}) & (\num{0.000})\\
public & \num{-0.007} & \num{-0.006}\\
 & (\num{0.029}) & (\num{0.029})\\
protest & \num{0.075} & \num{0.038}\\
 & (\num{0.034}) & (\num{0.035})\\
merged &  & \num{-0.009}\\
 &  & (\num{0.006})\\
avgfem &  & \num{0.076}\\
 &  & (\num{0.099})\\
avgage &  & \num{-0.018}***\\
 &  & (\num{0.004})\\
avglowgr &  & \num{-0.071}\\
 &  & (\num{0.096})\\
\midrule
Num.Obs. & \num{6520} & \num{6520}\\
R2 & \num{0.007} & \num{0.012}\\
AIC & \num{99424.7} & \num{99398.2}\\
Log.Lik. & \num{-49704.363} & \num{-49687.124}\\
concordance & \num{0.528} & \num{0.541}\\
n & \num{6520.000} & \num{6520.000}\\
nevent & \num{6324.000} & \num{6324.000}\\
p.value.log & \num{0.000} & \num{0.000}\\
p.value.robust & \num{0.000} & \num{0.000}\\
p.value.sc & \num{0.000} & \num{0.000}\\
p.value.wald & \num{0.000} & \num{0.000}\\
r.squared.max & \num{1.000} & \num{1.000}\\
statistic.log & \num{45.791} & \num{80.269}\\
statistic.robust & \num{30.195} & \num{39.423}\\
statistic.sc & \num{47.703} & \num{81.959}\\
statistic.wald & \num{34.180} & \num{47.160}\\
std.error.concordance & \num{0.006} & \num{0.007}\\
\bottomrule
\multicolumn{3}{l}{\rule{0pt}{1em}* p $<$ 0.1, ** p $<$ 0.05, *** p $<$ 0.01}\\
\end{tabular}
\end{table}

Comparing the most elaborate specification of the Cox model with the
most elaborate PWC model, we observe an interesting finding. That is,
the model estimates appear to be very sensitive to our choice of model.
For example, gender appears to be significant at the 1\% level for the
PWC model, with a positive sign, and for the Cox model it has the same
significance level, but has a negative sign. This clearly illustrates
the sensitivity of the parameter estimates to the parametric form
imposed.

\begin{enumerate}
\def\labelenumi{\arabic{enumi}.}
\setcounter{enumi}{4}
\tightlist
\item
  Repeat the procedure of question 2 for a Weibull model with (e.g.,
  gamma) unobserved heterogeneity. Compare the estimates of the
  regression coefficients across the models with and without unobserved
  heterogeneity.
\end{enumerate}

As above, but now with a Piecewise Constant (PWC) specification, where
you have an elaborate specification of the baseline hazard (say, 20
dummies).

Compare the estimates of the Cox model (question 4) with the results of
the PWC model with unobserved heterogeneity.

\begin{enumerate}
\def\labelenumi{\arabic{enumi}.}
\setcounter{enumi}{5}
\tightlist
\item
  Multiple Spells
\end{enumerate}

Estimate a standard Cox model (PL) and estimate Stratified Cox models
(SPL). Concerning the latter, estimate SPL models, where the school is
the stratum and estimate one where the teacher is the stratum. Comment
on the teacher SPL approach. Compare the PL and the school SPL
estimates. Can you think of a test to test for the relevance of using
the school SPL (rather than doing the PL)?

\begin{table}[!h]

\caption{\label{tab:unnamed-chunk-25}Stratified Cox Model on Schools}
\centering
\fontsize{8}{10}\selectfont
\begin{tabular}[t]{lcc}
\toprule
  & Model 1 & Model 2\\
\midrule
marstat & \num{0.017} & \num{0.018}\\
 & (\num{0.028}) & (\num{0.028})\\
gender & \num{-0.152}*** & \num{-0.152}***\\
 & (\num{0.037}) & (\num{0.037})\\
contract & \num{0.193}*** & \num{0.192}***\\
 & (\num{0.058}) & (\num{0.058})\\
lowgroup & \num{-0.027} & \num{-0.026}\\
 & (\num{0.034}) & (\num{0.034})\\
classize & \num{-0.001} & \num{-0.001}\\
 & (\num{0.002}) & (\num{0.002})\\
schsize & \num{0.000} & \num{0.000}\\
 & (\num{0.001}) & (\num{0.001})\\
public & \num{-0.125} & \num{-0.123}\\
 & (\num{0.140}) & (\num{0.140})\\
protest & \num{-0.202} & \num{-0.205}\\
 & (\num{0.169}) & (\num{0.169})\\
merged &  & \num{-0.020}**\\
 &  & (\num{0.010})\\
avgfem &  & \num{NA}\\
 &  & \vphantom{2} (\num{0.000})\\
avgage &  & \num{NA}\\
 &  & \vphantom{1} (\num{0.000})\\
avglowgr &  & \num{NA}\\
 &  & (\num{0.000})\\
\midrule
Num.Obs. & \num{6520} & \num{6520}\\
R2 & \num{0.006} & \num{0.006}\\
AIC & \num{29127.5} & \num{29125.3}\\
Log.Lik. & \num{-14555.767} & \num{-14553.668}\\
concordance & \num{0.516} & \num{0.523}\\
n & \num{6520.000} & \num{6520.000}\\
nevent & \num{6324.000} & \num{6324.000}\\
p.value.log & \num{0.000} & \num{0.000}\\
p.value.robust & \num{0.000} & \num{0.000}\\
p.value.sc & \num{0.000} & \num{0.000}\\
p.value.wald & \num{0.000} & \num{0.000}\\
r.squared.max & \num{0.989} & \num{0.989}\\
statistic.log & \num{37.798} & \num{41.998}\\
statistic.robust & \num{29.205} & \num{31.068}\\
statistic.sc & \num{38.376} & \num{42.529}\\
statistic.wald & \num{33.260} & \num{35.670}\\
std.error.concordance & \num{0.012} & \num{0.012}\\
\bottomrule
\multicolumn{3}{l}{\rule{0pt}{1em}* p $<$ 0.1, ** p $<$ 0.05, *** p $<$ 0.01}\\
\end{tabular}
\end{table}

Estimate a model with school specific dummies and compare these
estimates with those obtained from the school SPL.

\begin{table}[!h]

\caption{\label{tab:unnamed-chunk-26}Cox Model with School Dummies}
\centering
\fontsize{8}{10}\selectfont
\begin{tabular}[t]{lcc}
\toprule
  & Model 1 & Model 2\\
\midrule
schooldummies & \num{0.001}*** & \num{0.001}***\\
 & (\num{0.000}) & (\num{0.000})\\
marstat & \num{-0.092} & \num{-0.079}\\
 & (\num{0.129}) & (\num{0.129})\\
gender & \num{-0.316}* & \num{-0.392}**\\
 & (\num{0.182}) & (\num{0.192})\\
contract & \num{-0.419} & \num{-0.445}\\
 & (\num{0.706}) & (\num{0.711})\\
lowgroup & \num{0.094} & \num{0.110}\\
 & (\num{0.172}) & (\num{0.179})\\
classize & \num{0.003} & \num{0.004}\\
 & (\num{0.008}) & (\num{0.008})\\
schsize & \num{0.000} & \num{-0.001}\\
 & (\num{0.001}) & (\num{0.001})\\
public & \num{0.027} & \num{0.006}\\
 & (\num{0.166}) & (\num{0.169})\\
protest & \num{-0.002} & \num{-0.036}\\
 & (\num{0.196}) & (\num{0.202})\\
merged &  & \num{-0.008}\\
 &  & (\num{0.036})\\
avgfem &  & \num{0.559}\\
 &  & (\num{0.571})\\
avgage &  & \num{-0.011}\\
 &  & (\num{0.020})\\
avglowgr &  & \num{0.038}\\
 &  & (\num{0.567})\\
\midrule
Num.Obs. & \num{6520} & \num{6520}\\
R2 & \num{0.002} & \num{0.002}\\
AIC & \num{2051.4} & \num{2057.4}\\
Log.Lik. & \num{-1016.708} & \num{-1015.698}\\
concordance & \num{0.604} & \num{0.633}\\
n & \num{6520.000} & \num{6520.000}\\
nevent & \num{196.000} & \num{196.000}\\
p.value.log & \num{0.119} & \num{0.243}\\
p.value.robust & \num{0.243} & \num{0.392}\\
p.value.sc & \num{0.100} & \num{0.207}\\
p.value.wald & \num{0.152} & \num{0.291}\\
r.squared.max & \num{0.270} & \num{0.270}\\
statistic.log & \num{14.099} & \num{16.119}\\
statistic.robust & \num{11.493} & \num{13.745}\\
statistic.sc & \num{14.669} & \num{16.839}\\
statistic.wald & \num{13.250} & \num{15.270}\\
std.error.concordance & \num{0.039} & \num{0.038}\\
\bottomrule
\multicolumn{3}{l}{\rule{0pt}{1em}* p $<$ 0.1, ** p $<$ 0.05, *** p $<$ 0.01}\\
\end{tabular}
\end{table}

Observed sickness patterns vary between schools. This may be due to
sorting effects (bad teachers are the reason why the school scores bad
in absenteeism) and/or the school effects (it is elements of the school
that make some schools worse than others. Can you think of a
test/procedure to shed some more light on this issue?

\end{document}

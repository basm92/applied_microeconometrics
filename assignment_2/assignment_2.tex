% Options for packages loaded elsewhere
\PassOptionsToPackage{unicode}{hyperref}
\PassOptionsToPackage{hyphens}{url}
%
\documentclass[
]{article}
\usepackage{lmodern}
\usepackage{amssymb,amsmath}
\usepackage{ifxetex,ifluatex}
\ifnum 0\ifxetex 1\fi\ifluatex 1\fi=0 % if pdftex
  \usepackage[T1]{fontenc}
  \usepackage[utf8]{inputenc}
  \usepackage{textcomp} % provide euro and other symbols
\else % if luatex or xetex
  \usepackage{unicode-math}
  \defaultfontfeatures{Scale=MatchLowercase}
  \defaultfontfeatures[\rmfamily]{Ligatures=TeX,Scale=1}
\fi
% Use upquote if available, for straight quotes in verbatim environments
\IfFileExists{upquote.sty}{\usepackage{upquote}}{}
\IfFileExists{microtype.sty}{% use microtype if available
  \usepackage[]{microtype}
  \UseMicrotypeSet[protrusion]{basicmath} % disable protrusion for tt fonts
}{}
\makeatletter
\@ifundefined{KOMAClassName}{% if non-KOMA class
  \IfFileExists{parskip.sty}{%
    \usepackage{parskip}
  }{% else
    \setlength{\parindent}{0pt}
    \setlength{\parskip}{6pt plus 2pt minus 1pt}}
}{% if KOMA class
  \KOMAoptions{parskip=half}}
\makeatother
\usepackage{xcolor}
\IfFileExists{xurl.sty}{\usepackage{xurl}}{} % add URL line breaks if available
\IfFileExists{bookmark.sty}{\usepackage{bookmark}}{\usepackage{hyperref}}
\hypersetup{
  pdftitle={Applied Microeconometrics - Assignment 2},
  pdfauthor={Walter Verwer (589962) \& Bas Machielsen (590049)},
  hidelinks,
  pdfcreator={LaTeX via pandoc}}
\urlstyle{same} % disable monospaced font for URLs
\usepackage[margin=1in]{geometry}
\usepackage{color}
\usepackage{fancyvrb}
\newcommand{\VerbBar}{|}
\newcommand{\VERB}{\Verb[commandchars=\\\{\}]}
\DefineVerbatimEnvironment{Highlighting}{Verbatim}{commandchars=\\\{\}}
% Add ',fontsize=\small' for more characters per line
\usepackage{framed}
\definecolor{shadecolor}{RGB}{248,248,248}
\newenvironment{Shaded}{\begin{snugshade}}{\end{snugshade}}
\newcommand{\AlertTok}[1]{\textcolor[rgb]{0.94,0.16,0.16}{#1}}
\newcommand{\AnnotationTok}[1]{\textcolor[rgb]{0.56,0.35,0.01}{\textbf{\textit{#1}}}}
\newcommand{\AttributeTok}[1]{\textcolor[rgb]{0.77,0.63,0.00}{#1}}
\newcommand{\BaseNTok}[1]{\textcolor[rgb]{0.00,0.00,0.81}{#1}}
\newcommand{\BuiltInTok}[1]{#1}
\newcommand{\CharTok}[1]{\textcolor[rgb]{0.31,0.60,0.02}{#1}}
\newcommand{\CommentTok}[1]{\textcolor[rgb]{0.56,0.35,0.01}{\textit{#1}}}
\newcommand{\CommentVarTok}[1]{\textcolor[rgb]{0.56,0.35,0.01}{\textbf{\textit{#1}}}}
\newcommand{\ConstantTok}[1]{\textcolor[rgb]{0.00,0.00,0.00}{#1}}
\newcommand{\ControlFlowTok}[1]{\textcolor[rgb]{0.13,0.29,0.53}{\textbf{#1}}}
\newcommand{\DataTypeTok}[1]{\textcolor[rgb]{0.13,0.29,0.53}{#1}}
\newcommand{\DecValTok}[1]{\textcolor[rgb]{0.00,0.00,0.81}{#1}}
\newcommand{\DocumentationTok}[1]{\textcolor[rgb]{0.56,0.35,0.01}{\textbf{\textit{#1}}}}
\newcommand{\ErrorTok}[1]{\textcolor[rgb]{0.64,0.00,0.00}{\textbf{#1}}}
\newcommand{\ExtensionTok}[1]{#1}
\newcommand{\FloatTok}[1]{\textcolor[rgb]{0.00,0.00,0.81}{#1}}
\newcommand{\FunctionTok}[1]{\textcolor[rgb]{0.00,0.00,0.00}{#1}}
\newcommand{\ImportTok}[1]{#1}
\newcommand{\InformationTok}[1]{\textcolor[rgb]{0.56,0.35,0.01}{\textbf{\textit{#1}}}}
\newcommand{\KeywordTok}[1]{\textcolor[rgb]{0.13,0.29,0.53}{\textbf{#1}}}
\newcommand{\NormalTok}[1]{#1}
\newcommand{\OperatorTok}[1]{\textcolor[rgb]{0.81,0.36,0.00}{\textbf{#1}}}
\newcommand{\OtherTok}[1]{\textcolor[rgb]{0.56,0.35,0.01}{#1}}
\newcommand{\PreprocessorTok}[1]{\textcolor[rgb]{0.56,0.35,0.01}{\textit{#1}}}
\newcommand{\RegionMarkerTok}[1]{#1}
\newcommand{\SpecialCharTok}[1]{\textcolor[rgb]{0.00,0.00,0.00}{#1}}
\newcommand{\SpecialStringTok}[1]{\textcolor[rgb]{0.31,0.60,0.02}{#1}}
\newcommand{\StringTok}[1]{\textcolor[rgb]{0.31,0.60,0.02}{#1}}
\newcommand{\VariableTok}[1]{\textcolor[rgb]{0.00,0.00,0.00}{#1}}
\newcommand{\VerbatimStringTok}[1]{\textcolor[rgb]{0.31,0.60,0.02}{#1}}
\newcommand{\WarningTok}[1]{\textcolor[rgb]{0.56,0.35,0.01}{\textbf{\textit{#1}}}}
\usepackage{longtable,booktabs}
% Correct order of tables after \paragraph or \subparagraph
\usepackage{etoolbox}
\makeatletter
\patchcmd\longtable{\par}{\if@noskipsec\mbox{}\fi\par}{}{}
\makeatother
% Allow footnotes in longtable head/foot
\IfFileExists{footnotehyper.sty}{\usepackage{footnotehyper}}{\usepackage{footnote}}
\makesavenoteenv{longtable}
\usepackage{graphicx}
\makeatletter
\def\maxwidth{\ifdim\Gin@nat@width>\linewidth\linewidth\else\Gin@nat@width\fi}
\def\maxheight{\ifdim\Gin@nat@height>\textheight\textheight\else\Gin@nat@height\fi}
\makeatother
% Scale images if necessary, so that they will not overflow the page
% margins by default, and it is still possible to overwrite the defaults
% using explicit options in \includegraphics[width, height, ...]{}
\setkeys{Gin}{width=\maxwidth,height=\maxheight,keepaspectratio}
% Set default figure placement to htbp
\makeatletter
\def\fps@figure{htbp}
\makeatother
\setlength{\emergencystretch}{3em} % prevent overfull lines
\providecommand{\tightlist}{%
  \setlength{\itemsep}{0pt}\setlength{\parskip}{0pt}}
\setcounter{secnumdepth}{-\maxdimen} % remove section numbering
\let\oldShaded\Shaded
\let\endoldShaded\endShaded
\renewenvironment{Shaded}{\footnotesize\oldShaded}{\endoldShaded}
\usepackage{booktabs}
\usepackage{longtable}
\usepackage{array}
\usepackage{multirow}
\usepackage{wrapfig}
\usepackage{float}
\usepackage{colortbl}
\usepackage{pdflscape}
\usepackage{tabu}
\usepackage{threeparttable}
\usepackage{threeparttablex}
\usepackage[normalem]{ulem}
\usepackage{makecell}
\usepackage{xcolor}
\usepackage{siunitx}
\newcolumntype{d}{S[input-symbols = ()]}
\ifluatex
  \usepackage{selnolig}  % disable illegal ligatures
\fi

\title{Applied Microeconometrics - Assignment 2}
\author{Walter Verwer (589962) \& Bas Machielsen (590049)}
\date{\today}

\begin{document}
\maketitle

\begin{enumerate}
\def\labelenumi{\arabic{enumi}.}
\tightlist
\item
  Compute the average probability to receive benefits 10 and 30 weeks
  after application for applicants that had a search period and
  applicants that did not have a search period.
\end{enumerate}

\begin{Shaded}
\begin{Highlighting}[]
\NormalTok{dataset }\SpecialCharTok{\%\textgreater{}\%}
    \FunctionTok{group\_by}\NormalTok{(searchperiod) }\SpecialCharTok{\%\textgreater{}\%}
    \FunctionTok{summarize}\NormalTok{(}\AttributeTok{prob\_10weeks =} \FunctionTok{mean}\NormalTok{(benefits\_week10), }\AttributeTok{prob\_30weeks =} \FunctionTok{mean}\NormalTok{(benefits\_week30)) }\SpecialCharTok{\%\textgreater{}\%}
\NormalTok{    knitr}\SpecialCharTok{::}\FunctionTok{kable}\NormalTok{()}
\end{Highlighting}
\end{Shaded}

\begin{longtable}[]{@{}rrr@{}}
\toprule
searchperiod & prob\_10weeks & prob\_30weeks\tabularnewline
\midrule
\endhead
0 & 0.7359116 & 0.5403315\tabularnewline
1 & 0.5723684 & 0.4144737\tabularnewline
\bottomrule
\end{longtable}

It seems that there is a large difference in unconditional means in the
outcome variable among treated and controlled groups. Individuals
exposed to the treatment (a search period) have much lower probabilities
of ultimately receiving benefits, whether this is after 10 weeks, or
after 30 weeks. This could be a potential indication of the presence of
a treatment effect, but a more rigorous examination should ensue.

\begin{enumerate}
\def\labelenumi{\arabic{enumi}.}
\setcounter{enumi}{1}
\tightlist
\item
  Make a balancing table in which you compare characteristics of
  applicants with and without a search period.
\end{enumerate}

\begin{Shaded}
\begin{Highlighting}[]
\NormalTok{modelsummary}\SpecialCharTok{::}\FunctionTok{datasummary\_balance}\NormalTok{(}\SpecialCharTok{\textasciitilde{}}\NormalTok{ searchperiod,}
                                  \AttributeTok{data =}\NormalTok{ dataset }\SpecialCharTok{\%\textgreater{}\%}
                                      \FunctionTok{select}\NormalTok{(}\FunctionTok{c}\NormalTok{(}\DecValTok{1}\NormalTok{,}\DecValTok{4}\SpecialCharTok{:}\DecValTok{23}\NormalTok{)) }\SpecialCharTok{\%\textgreater{}\%}
                                      \FunctionTok{mutate}\NormalTok{(}\AttributeTok{searchperiod =} \FunctionTok{if\_else}\NormalTok{(}
\NormalTok{                                          searchperiod }\SpecialCharTok{==} \DecValTok{1}\NormalTok{,}
                                          \StringTok{"With Search"}\NormalTok{, }
                                          \StringTok{"Without Search"}\NormalTok{)), }
                                  \AttributeTok{output =} \StringTok{"latex"}\NormalTok{, }
                                  \AttributeTok{fmt =} \StringTok{"\%.3f"}\NormalTok{,}
                                  \AttributeTok{dinm =} \ConstantTok{TRUE}\NormalTok{,}
                                  \AttributeTok{dinm\_statistic =} \StringTok{"p.value"}
\NormalTok{                                    ) }\SpecialCharTok{\%\textgreater{}\%}
\NormalTok{    kableExtra}\SpecialCharTok{::}\FunctionTok{kable\_styling}\NormalTok{(}\AttributeTok{font\_size =} \DecValTok{10}\NormalTok{)}
\end{Highlighting}
\end{Shaded}

\begin{verbatim}
## Warning: To compile a LaTeX document with this table, the following commands must be placed in the document preamble:
## 
## \usepackage{booktabs}
## \usepackage{siunitx}
## \newcolumntype{d}{S[input-symbols = ()]}
## 
## To disable `siunitx` and prevent `modelsummary` from wrapping numeric entries in `\num{}`, call:
## 
## options("modelsummary_format_numeric_latex" = "plain")
## 
## This warning is displayed once per session.
\end{verbatim}

\begin{verbatim}
## Warning in modelsummary::datasummary_balance(~searchperiod, data = dataset %>% : Please install the `estimatr` package or set `dinm=FALSE` to
##              suppress this warning.
\end{verbatim}

\begin{table}
\centering
\fontsize{10}{12}\selectfont
\begin{tabular}[t]{lrrrr}
\toprule
\multicolumn{1}{c}{ } & \multicolumn{2}{c}{With Search (N=760)} & \multicolumn{2}{c}{Without Search (N=905)} \\
\cmidrule(l{3pt}r{3pt}){2-3} \cmidrule(l{3pt}r{3pt}){4-5}
  & Mean & Std. Dev. & Mean  & Std. Dev. \\
\midrule
sumincome_12monthsbefore & 1.259 & 1.099 & 1.296 & 1.052\\
sumincome_24monthsbefore & 2.689 & 2.125 & 2.785 & 2.054\\
age & 37.259 & 8.657 & 39.926 & 9.031\\
female & 0.372 & 0.484 & 0.397 & 0.490\\
children & 0.114 & 0.319 & 0.164 & 0.370\\
partner & 0.107 & 0.309 & 0.126 & 0.332\\
period1 & 0.222 & 0.416 & 0.264 & 0.441\\
period2 & 0.233 & 0.423 & 0.256 & 0.437\\
period3 & 0.286 & 0.452 & 0.265 & 0.442\\
period4 & 0.259 & 0.438 & 0.214 & 0.411\\
location1 & 0.113 & 0.317 & 0.177 & 0.382\\
location2 & 0.232 & 0.422 & 0.182 & 0.386\\
location3 & 0.300 & 0.459 & 0.373 & 0.484\\
location4 & 0.222 & 0.416 & 0.101 & 0.301\\
location5 & 0.133 & 0.340 & 0.167 & 0.373\\
educ_bachelormaster & 0.267 & 0.443 & 0.264 & 0.441\\
educ_prepvocational & 0.200 & 0.400 & 0.218 & 0.413\\
educ_primaryorless & 0.149 & 0.356 & 0.130 & 0.337\\
educ_unknown & 0.050 & 0.218 & 0.014 & 0.119\\
educ_vocational & 0.334 & 0.472 & 0.373 & 0.484\\
\bottomrule
\end{tabular}
\end{table}

It seems that all covariates are rather balanced, indicated by the
absence of significant differences in means among the treated and the
control group. Of course, because we are dealing with a large number of
joint null-hypotheses, we should only reject the null hypothesis
according to a Bonferroni-corrected p-value. If our regular p-value
criterion would be \(p < 0.05\), in this case, we reject the null
hypothesis when \(p < \frac{0.05}{20} = 0.0025\). Even with this
criterion, most of the location dummies are still significantly
different in treatment and control groups, indicating that perhaps the
treatment was administered in different regions, but was stratified
according to all other observables. Adding region-specific fixed effects
to the regression specifications should solve this problem.

\begin{enumerate}
\def\labelenumi{\arabic{enumi}.}
\setcounter{enumi}{2}
\tightlist
\item
  Regress the outcome variables first only on whether or not a search
  period was applied (which should give the difference-in-means
  estimate) and next include other covariates in the regression.
\end{enumerate}

\begin{Shaded}
\begin{Highlighting}[]
\NormalTok{model1 }\OtherTok{\textless{}{-}} \FunctionTok{lm}\NormalTok{(}\AttributeTok{data =}\NormalTok{ dataset, }\AttributeTok{formula =}\NormalTok{ benefits\_week10 }\SpecialCharTok{\textasciitilde{}}\NormalTok{ searchperiod)}
\NormalTok{model2 }\OtherTok{\textless{}{-}} \FunctionTok{lm}\NormalTok{(}\AttributeTok{data =}\NormalTok{ dataset, }\AttributeTok{formula =}\NormalTok{ benefits\_week30 }\SpecialCharTok{\textasciitilde{}}\NormalTok{ searchperiod)}
\NormalTok{model3 }\OtherTok{\textless{}{-}} \FunctionTok{update}\NormalTok{(model1, . }\SpecialCharTok{\textasciitilde{}}\NormalTok{ . }\SpecialCharTok{+}\NormalTok{ period1 }\SpecialCharTok{+}\NormalTok{ period2 }\SpecialCharTok{+}\NormalTok{ period3 }\SpecialCharTok{+}\NormalTok{ period4 }\SpecialCharTok{+} 
\NormalTok{                     location1 }\SpecialCharTok{+}\NormalTok{ location2 }\SpecialCharTok{+}\NormalTok{ location3 }\SpecialCharTok{+}\NormalTok{ location4)}
\NormalTok{model4 }\OtherTok{\textless{}{-}} \FunctionTok{update}\NormalTok{(model2, . }\SpecialCharTok{\textasciitilde{}}\NormalTok{ . }\SpecialCharTok{+}\NormalTok{ period1 }\SpecialCharTok{+}\NormalTok{ period2 }\SpecialCharTok{+}\NormalTok{ period3 }\SpecialCharTok{+}\NormalTok{ period4 }\SpecialCharTok{+} 
\NormalTok{                     location1 }\SpecialCharTok{+}\NormalTok{ location2 }\SpecialCharTok{+}\NormalTok{ location3 }\SpecialCharTok{+}\NormalTok{ location4)}
\NormalTok{model5 }\OtherTok{\textless{}{-}} \FunctionTok{update}\NormalTok{(model3, . }\SpecialCharTok{\textasciitilde{}}\NormalTok{ . }\SpecialCharTok{+}\NormalTok{ sumincome\_12monthsbefore }\SpecialCharTok{+} 
\NormalTok{                     sumincome\_24monthsbefore }\SpecialCharTok{+}\NormalTok{ age }\SpecialCharTok{+}\NormalTok{ female }\SpecialCharTok{+}\NormalTok{ children }\SpecialCharTok{+} 
\NormalTok{                     partner }\SpecialCharTok{+}\NormalTok{ educ\_bachelormaster }\SpecialCharTok{+}\NormalTok{ educ\_prepvocational }\SpecialCharTok{+} 
\NormalTok{                     educ\_primaryorless }\SpecialCharTok{+}\NormalTok{ educ\_unknown }\SpecialCharTok{+}\NormalTok{ educ\_vocational)}
\NormalTok{model6 }\OtherTok{\textless{}{-}} \FunctionTok{update}\NormalTok{(model4, . }\SpecialCharTok{\textasciitilde{}}\NormalTok{ . }\SpecialCharTok{+}\NormalTok{ sumincome\_12monthsbefore }\SpecialCharTok{+} 
\NormalTok{                     sumincome\_24monthsbefore }\SpecialCharTok{+}\NormalTok{ age }\SpecialCharTok{+}\NormalTok{ female }\SpecialCharTok{+}\NormalTok{ children }\SpecialCharTok{+} 
\NormalTok{                     partner }\SpecialCharTok{+}\NormalTok{ educ\_bachelormaster }\SpecialCharTok{+}\NormalTok{ educ\_prepvocational }\SpecialCharTok{+}
\NormalTok{                     educ\_primaryorless }\SpecialCharTok{+}\NormalTok{ educ\_unknown }\SpecialCharTok{+}\NormalTok{ educ\_vocational)}

\NormalTok{models }\OtherTok{\textless{}{-}} \FunctionTok{list}\NormalTok{(model1, model2, model3, model4, model5, model6)}
\end{Highlighting}
\end{Shaded}

\begin{Shaded}
\begin{Highlighting}[]
\FunctionTok{stargazer}\NormalTok{(models, }\AttributeTok{title =} \StringTok{"Estimations of the Effect of Search on P(Benefits)"}\NormalTok{,}
          \AttributeTok{label =} \StringTok{"tab:reg"}\NormalTok{, }\AttributeTok{header=}\ConstantTok{FALSE}\NormalTok{, }\AttributeTok{model.names =} \ConstantTok{FALSE}\NormalTok{,}
          \AttributeTok{column.sep.width=}\StringTok{"0pt"}\NormalTok{,}
          \AttributeTok{df=}\NormalTok{F,}
          \AttributeTok{dep.var.labels =} \FunctionTok{c}\NormalTok{(}\FunctionTok{rep}\NormalTok{(}\StringTok{"Benefits"}\NormalTok{,}\DecValTok{6}\NormalTok{)),}
          \AttributeTok{column.labels=} \FunctionTok{c}\NormalTok{(}\FunctionTok{rep}\NormalTok{(}\FunctionTok{c}\NormalTok{(}\StringTok{"10 Weeks"}\NormalTok{, }\StringTok{"30 Weeks"}\NormalTok{),}\DecValTok{3}\NormalTok{)),}
          \AttributeTok{omit =} \FunctionTok{c}\NormalTok{(}\StringTok{"period1"}\NormalTok{, }\StringTok{"period2"}\NormalTok{, }\StringTok{"period3"}\NormalTok{, }\StringTok{"period4"}\NormalTok{,}\StringTok{"location"}\NormalTok{),}
          \AttributeTok{add.lines =} \FunctionTok{list}\NormalTok{(}\FunctionTok{c}\NormalTok{(}\StringTok{"Period Dummies"}\NormalTok{, }\FunctionTok{rep}\NormalTok{(}\StringTok{"No"}\NormalTok{, }\DecValTok{2}\NormalTok{), }\FunctionTok{rep}\NormalTok{(}\StringTok{"Yes"}\NormalTok{, }\DecValTok{4}\NormalTok{)),}
                            \FunctionTok{c}\NormalTok{(}\StringTok{"Region Dummies"}\NormalTok{, }\FunctionTok{rep}\NormalTok{(}\StringTok{"No"}\NormalTok{, }\DecValTok{2}\NormalTok{), }\FunctionTok{rep}\NormalTok{(}\StringTok{"Yes"}\NormalTok{, }\DecValTok{4}\NormalTok{))),}
          \AttributeTok{omit.stat =} \FunctionTok{c}\NormalTok{(}\StringTok{"ll"}\NormalTok{, }\StringTok{"ser"}\NormalTok{, }\StringTok{"rsq"}\NormalTok{))}
\end{Highlighting}
\end{Shaded}

\begin{table}[!htbp] \centering 
  \caption{Estimations of the Effect of Search on P(Benefits)} 
  \label{tab:reg} 
\begin{tabular}{@{\extracolsep{0pt}}lcccccc} 
\\[-1.8ex]\hline 
\hline \\[-1.8ex] 
 & \multicolumn{6}{c}{\textit{Dependent variable:}} \\ 
\cline{2-7} 
\\[-1.8ex] & Benefits & Benefits & Benefits & Benefits & Benefits & Benefits \\ 
 & 10 Weeks & 30 Weeks & 10 Weeks & 30 Weeks & 10 Weeks & 30 Weeks \\ 
\\[-1.8ex] & (1) & (2) & (3) & (4) & (5) & (6)\\ 
\hline \\[-1.8ex] 
 searchperiod & $-$0.164$^{***}$ & $-$0.126$^{***}$ & $-$0.157$^{***}$ & $-$0.121$^{***}$ & $-$0.143$^{***}$ & $-$0.099$^{***}$ \\ 
  & (0.023) & (0.024) & (0.023) & (0.025) & (0.024) & (0.025) \\ 
  & & & & & & \\ 
 sumincome\_12monthsbefore &  &  &  &  & 0.0004 & $-$0.022 \\ 
  &  &  &  &  & (0.027) & (0.028) \\ 
  & & & & & & \\ 
 sumincome\_24monthsbefore &  &  &  &  & $-$0.009 & $-$0.005 \\ 
  &  &  &  &  & (0.014) & (0.014) \\ 
  & & & & & & \\ 
 age &  &  &  &  & 0.001 & 0.004$^{***}$ \\ 
  &  &  &  &  & (0.001) & (0.001) \\ 
  & & & & & & \\ 
 female &  &  &  &  & $-$0.010 & $-$0.028 \\ 
  &  &  &  &  & (0.024) & (0.026) \\ 
  & & & & & & \\ 
 children &  &  &  &  & $-$0.037 & 0.002 \\ 
  &  &  &  &  & (0.037) & (0.040) \\ 
  & & & & & & \\ 
 partner &  &  &  &  & 0.056 & 0.078$^{*}$ \\ 
  &  &  &  &  & (0.040) & (0.043) \\ 
  & & & & & & \\ 
 educ\_bachelormaster &  &  &  &  & $-$0.092$^{***}$ & $-$0.116$^{***}$ \\ 
  &  &  &  &  & (0.029) & (0.031) \\ 
  & & & & & & \\ 
 educ\_prepvocational &  &  &  &  & 0.013 & 0.022 \\ 
  &  &  &  &  & (0.032) & (0.033) \\ 
  & & & & & & \\ 
 educ\_primaryorless &  &  &  &  & $-$0.034 & 0.033 \\ 
  &  &  &  &  & (0.037) & (0.039) \\ 
  & & & & & & \\ 
 educ\_unknown &  &  &  &  & $-$0.381$^{***}$ & $-$0.270$^{***}$ \\ 
  &  &  &  &  & (0.068) & (0.072) \\ 
  & & & & & & \\ 
 educ\_vocational &  &  &  &  &  &  \\ 
  &  &  &  &  &  &  \\ 
  & & & & & & \\ 
 Constant & 0.736$^{***}$ & 0.540$^{***}$ & 0.682$^{***}$ & 0.404$^{***}$ & 0.723$^{***}$ & 0.326$^{***}$ \\ 
  & (0.016) & (0.016) & (0.038) & (0.040) & (0.068) & (0.072) \\ 
  & & & & & & \\ 
\hline \\[-1.8ex] 
Period Dummies & No & No & Yes & Yes & Yes & Yes \\ 
Region Dummies & No & No & Yes & Yes & Yes & Yes \\ 
Observations & 1,665 & 1,665 & 1,665 & 1,665 & 1,663 & 1,663 \\ 
Adjusted R$^{2}$ & 0.029 & 0.015 & 0.034 & 0.020 & 0.057 & 0.054 \\ 
F Statistic & 50.771$^{***}$ & 26.592$^{***}$ & 8.301$^{***}$ & 5.298$^{***}$ & 6.565$^{***}$ & 6.304$^{***}$ \\ 
\hline 
\hline \\[-1.8ex] 
\textit{Note:}  & \multicolumn{6}{r}{$^{*}$p$<$0.1; $^{**}$p$<$0.05; $^{***}$p$<$0.01} \\ 
\end{tabular} 
\end{table}

The results imply that the treatment is effective in reducing by
10-percentage points the probability of receiving benefits on the
long-term (30 weeks), and slightly higher (15 percentage points) on the
short-term (10-weeks). If there is no selection on unobservables, these
estimates give a good estimate of the ATE. But to what extent can these
estimates be trusted?

\clearpage

\begin{enumerate}
\def\labelenumi{\arabic{enumi}.}
\setcounter{enumi}{3}
\tightlist
\item
  Compute the no-assumption bounds for the treatment effects.
\end{enumerate}

\begin{Shaded}
\begin{Highlighting}[]
\CommentTok{\# Implement the no assumption bounds}
\NormalTok{no\_assumption\_bounds }\OtherTok{\textless{}{-}} \ControlFlowTok{function}\NormalTok{(dataset, y\_min, y\_max, treatmentvar, depvar)\{}
\NormalTok{  depvar }\OtherTok{\textless{}{-}}\NormalTok{ dplyr}\SpecialCharTok{::}\FunctionTok{enquo}\NormalTok{(depvar)}
\NormalTok{  treatmentvar }\OtherTok{\textless{}{-}}\NormalTok{ dplyr}\SpecialCharTok{::}\FunctionTok{enquo}\NormalTok{(treatmentvar)}
  
\NormalTok{  pr\_treated }\OtherTok{\textless{}{-}}\NormalTok{ dataset }\SpecialCharTok{\%\textgreater{}\%}
    \FunctionTok{summarize}\NormalTok{(}\AttributeTok{mean =} \FunctionTok{mean}\NormalTok{(}\FunctionTok{UQ}\NormalTok{(treatmentvar), }\AttributeTok{na.rm =} \ConstantTok{TRUE}\NormalTok{)) }\SpecialCharTok{\%\textgreater{}\%}
    \FunctionTok{pull}\NormalTok{() }
  
\NormalTok{  pr\_untreated }\OtherTok{\textless{}{-}} \DecValTok{1}\SpecialCharTok{{-}}\NormalTok{pr\_treated}
  
\NormalTok{  expected\_y\_given\_deq1 }\OtherTok{\textless{}{-}}\NormalTok{ dataset }\SpecialCharTok{\%\textgreater{}\%}
\NormalTok{    dplyr}\SpecialCharTok{::}\FunctionTok{filter}\NormalTok{(}\FunctionTok{UQ}\NormalTok{(treatmentvar) }\SpecialCharTok{==} \DecValTok{1}\NormalTok{) }\SpecialCharTok{\%\textgreater{}\%}
             \FunctionTok{summarize}\NormalTok{(}\AttributeTok{mean =} \FunctionTok{mean}\NormalTok{(}\FunctionTok{UQ}\NormalTok{(depvar), }\AttributeTok{na.rm =} \ConstantTok{TRUE}\NormalTok{)) }\SpecialCharTok{\%\textgreater{}\%}
             \FunctionTok{pull}\NormalTok{()}
  
\NormalTok{  expected\_y\_given\_deq0 }\OtherTok{\textless{}{-}}\NormalTok{ dataset}\SpecialCharTok{\%\textgreater{}\%}
\NormalTok{    dplyr}\SpecialCharTok{::}\FunctionTok{filter}\NormalTok{(}\FunctionTok{UQ}\NormalTok{(treatmentvar) }\SpecialCharTok{==} \DecValTok{0}\NormalTok{) }\SpecialCharTok{\%\textgreater{}\%}
             \FunctionTok{summarize}\NormalTok{(}\AttributeTok{mean =} \FunctionTok{mean}\NormalTok{(}\FunctionTok{UQ}\NormalTok{(depvar), }\AttributeTok{na.rm =} \ConstantTok{TRUE}\NormalTok{)) }\SpecialCharTok{\%\textgreater{}\%}
             \FunctionTok{pull}\NormalTok{()}
           
  \CommentTok{\# bounds on y\^{}*\_1: }
\NormalTok{  lower\_bound\_y1 }\OtherTok{\textless{}{-}}\NormalTok{ expected\_y\_given\_deq1 }\SpecialCharTok{*}\NormalTok{ pr\_treated }\SpecialCharTok{+}\NormalTok{ y\_min }\SpecialCharTok{*}\NormalTok{ pr\_untreated }
\NormalTok{  upper\_bound\_y1 }\OtherTok{\textless{}{-}}\NormalTok{ expected\_y\_given\_deq1 }\SpecialCharTok{*}\NormalTok{ pr\_treated }\SpecialCharTok{+}\NormalTok{ y\_max }\SpecialCharTok{*}\NormalTok{ pr\_untreated}
  
  \CommentTok{\# bounds on y\^{}*\_0:}
\NormalTok{  lower\_bound\_y0 }\OtherTok{\textless{}{-}}\NormalTok{ expected\_y\_given\_deq0 }\SpecialCharTok{*}\NormalTok{ pr\_untreated }\SpecialCharTok{+}\NormalTok{ y\_min }\SpecialCharTok{*}\NormalTok{ pr\_treated }
\NormalTok{  upper\_bound\_y0 }\OtherTok{\textless{}{-}}\NormalTok{ expected\_y\_given\_deq0 }\SpecialCharTok{*}\NormalTok{ pr\_untreated }\SpecialCharTok{+}\NormalTok{ y\_max }\SpecialCharTok{*}\NormalTok{ pr\_treated}
  
  \CommentTok{\# bounds on the ATE:}
\NormalTok{  lower\_bound\_ate }\OtherTok{\textless{}{-}}\NormalTok{ expected\_y\_given\_deq1}\SpecialCharTok{*}\NormalTok{pr\_treated }\SpecialCharTok{{-}}\NormalTok{ expected\_y\_given\_deq0}\SpecialCharTok{*}\NormalTok{pr\_untreated }\SpecialCharTok{+} 
\NormalTok{    (y\_min }\SpecialCharTok{+}\NormalTok{ y\_max)}\SpecialCharTok{*}\NormalTok{pr\_untreated }\SpecialCharTok{{-}}\NormalTok{ y\_max }
\NormalTok{  upper\_bound\_ate }\OtherTok{\textless{}{-}}\NormalTok{ expected\_y\_given\_deq1}\SpecialCharTok{*}\NormalTok{pr\_treated }\SpecialCharTok{{-}}\NormalTok{ expected\_y\_given\_deq0}\SpecialCharTok{*}\NormalTok{pr\_untreated }\SpecialCharTok{+} 
\NormalTok{    (y\_min }\SpecialCharTok{+}\NormalTok{ y\_max)}\SpecialCharTok{*}\NormalTok{pr\_untreated }\SpecialCharTok{{-}}\NormalTok{ y\_min}
  
\NormalTok{  out }\OtherTok{\textless{}{-}} \FunctionTok{tribble}\NormalTok{(}\SpecialCharTok{\textasciitilde{}}\StringTok{"lower\_bound\_y1"}\NormalTok{, }\SpecialCharTok{\textasciitilde{}}\StringTok{"upper\_bound\_y1"}\NormalTok{, }\SpecialCharTok{\textasciitilde{}}\StringTok{"lower\_bound\_y0"}\NormalTok{,}
                 \SpecialCharTok{\textasciitilde{}}\StringTok{"upper\_bound\_y0"}\NormalTok{, }\SpecialCharTok{\textasciitilde{}}\StringTok{"lower\_bound\_ate"}\NormalTok{, }\SpecialCharTok{\textasciitilde{}}\StringTok{"upper\_bound\_ate"}\NormalTok{,}
\NormalTok{          lower\_bound\_y1, upper\_bound\_y1, lower\_bound\_y0, upper\_bound\_y0, lower\_bound\_ate, upper\_bound\_ate)}
  
  \FunctionTok{return}\NormalTok{(out)}
\NormalTok{\}}

\FunctionTok{no\_assumption\_bounds}\NormalTok{(dataset, }\DecValTok{0}\NormalTok{,}\DecValTok{1}\NormalTok{,searchperiod, benefits\_week10) }\SpecialCharTok{\%\textgreater{}\%}
\NormalTok{  knitr}\SpecialCharTok{::}\FunctionTok{kable}\NormalTok{(}\AttributeTok{booktabs=}\NormalTok{T) }\SpecialCharTok{\%\textgreater{}\%}
\NormalTok{    kableExtra}\SpecialCharTok{::}\FunctionTok{kable\_styling}\NormalTok{(}\AttributeTok{font\_size =} \DecValTok{7}\NormalTok{, }\AttributeTok{latex\_options =} \StringTok{"hold\_position"}\NormalTok{)}
\end{Highlighting}
\end{Shaded}

\begin{table}[!h]
\centering\begingroup\fontsize{7}{9}\selectfont

\begin{tabular}{rrrrrr}
\toprule
lower\_bound\_y1 & upper\_bound\_y1 & lower\_bound\_y0 & upper\_bound\_y0 & lower\_bound\_ate & upper\_bound\_ate\\
\midrule
0.2612613 & 0.8048048 & 0.4 & 0.8564565 & -0.5951952 & 0.4048048\\
\bottomrule
\end{tabular}
\endgroup{}
\end{table}

\begin{Shaded}
\begin{Highlighting}[]
\FunctionTok{no\_assumption\_bounds}\NormalTok{(dataset, }\DecValTok{0}\NormalTok{,}\DecValTok{1}\NormalTok{,searchperiod, benefits\_week30) }\SpecialCharTok{\%\textgreater{}\%}
\NormalTok{  knitr}\SpecialCharTok{::}\FunctionTok{kable}\NormalTok{(}\AttributeTok{booktabs=}\NormalTok{T) }\SpecialCharTok{\%\textgreater{}\%}
\NormalTok{    kableExtra}\SpecialCharTok{::}\FunctionTok{kable\_styling}\NormalTok{(}\AttributeTok{font\_size =} \DecValTok{7}\NormalTok{, }\AttributeTok{latex\_options =} \StringTok{"hold\_position"}\NormalTok{)}
\end{Highlighting}
\end{Shaded}

\begin{table}[!h]
\centering\begingroup\fontsize{7}{9}\selectfont

\begin{tabular}{rrrrrr}
\toprule
lower\_bound\_y1 & upper\_bound\_y1 & lower\_bound\_y0 & upper\_bound\_y0 & lower\_bound\_ate & upper\_bound\_ate\\
\midrule
0.1891892 & 0.7327327 & 0.2936937 & 0.7501502 & -0.560961 & 0.439039\\
\bottomrule
\end{tabular}
\endgroup{}
\end{table}

\begin{enumerate}
\def\labelenumi{\arabic{enumi}.}
\setcounter{enumi}{4}
\tightlist
\item
  Assume that caseworkers only apply search periods to applicants who
  benefit from it. How does this affects the bounds.
\end{enumerate}

If people only select into the treatment if it works (meaning,
decreasing the probability of benefits), we have:

\[
\mathbb{E}[Y^*_1 | D=1] \leq \mathbb{E}[Y^*_0 | D = 1] \text{ and } \mathbb{E}[Y^*_0 | D = 0] \leq \mathbb{E}[Y^*_1 | D = 0]
\] Since the case is the opposite of the case that is worked out on the
lecture slides, we cannot blindly apply the formulate, but realizing
that:

\[
y_{min} \leq \mathbb{E}[Y^*_1 | D=1] \leq \mathbb{E}[Y^*_0 | D = 1] \leq y_{max} \text{ and} \\
y_{min} \leq \mathbb{E}[Y^*_0 | D = 0] \leq \mathbb{E}[Y^*_1 | D = 0] \leq y_{max}
\] We can evaluate \(\mathbb{E}[Y^*_1]\), and we get:

\[
\mathbb{E}[Y^*_1 | D = 1] * \text{Pr}[D=1] + \text{Pr}[D=0] * \mathbb{E}[Y^*_0 | D=0] \leq \mathbb{E}[Y^*_1] \leq \mathbb{E}[Y^*_1 | D = 1] * \text{Pr}[D=1] + y_{max} * \text{Pr}[D=0]
\] And for \(\mathbb{E}[Y^*_0]\), we get:

\[
\mathbb{E}[Y^*_0 | D = 0] * \text{Pr}[D=0] + \text{Pr}[D=1] * \mathbb{E}[Y^*_1 | D = 1] \leq \mathbb{E}[Y^*_0] \leq \mathbb{E}[Y^*_0 | D = 0] * \text{Pr}[D=0] + \text{Pr}[D=1]*y_{max}
\] Then, realizing that the lower bound of \(\mathbb{E}[Y^*_1 - Y^*_0]\)
is the lower bound of \(\mathbb{E}[Y^*_1]\) minus the upper bound of
\(\mathbb{E}[Y^*_0]\), and \emph{mutatis mutandis} for the upper bound
of \(\mathbb{E}[Y^*_1 - Y^*_0]\), after rewriting, we find:

\[
-\text{Pr}(D=1) \cdot \left( y_{max} - \mathbb{E}[Y^*_1 | D = 1] \right) \leq \mathbb{E}[Y^*_1 - Y^*_0] \leq \text{Pr}(D=0) \cdot \left( y_{max} - \mathbb{E}[Y^*_0 | D = 0] \right)
\]

Which corresponds to the same properties as found in the lecture slides
(i.e.~narrower bounds, but without ever excluding zero).

\begin{enumerate}
\def\labelenumi{\arabic{enumi}.}
\setcounter{enumi}{5}
\item
  Next, imposed the monotone treatment response and the monotone
  treatment selection assumption separately and also jointly.
\item
  Usually higher educated workers have more favorable labor market
  outcomes. Use education as monotone instrumental variable and compute
  the bounds.
\end{enumerate}

\end{document}

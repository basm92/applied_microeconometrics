% Options for packages loaded elsewhere
\PassOptionsToPackage{unicode}{hyperref}
\PassOptionsToPackage{hyphens}{url}
%
\documentclass[
]{article}
\usepackage{lmodern}
\usepackage{amssymb,amsmath}
\usepackage{ifxetex,ifluatex}
\ifnum 0\ifxetex 1\fi\ifluatex 1\fi=0 % if pdftex
  \usepackage[T1]{fontenc}
  \usepackage[utf8]{inputenc}
  \usepackage{textcomp} % provide euro and other symbols
\else % if luatex or xetex
  \usepackage{unicode-math}
  \defaultfontfeatures{Scale=MatchLowercase}
  \defaultfontfeatures[\rmfamily]{Ligatures=TeX,Scale=1}
\fi
% Use upquote if available, for straight quotes in verbatim environments
\IfFileExists{upquote.sty}{\usepackage{upquote}}{}
\IfFileExists{microtype.sty}{% use microtype if available
  \usepackage[]{microtype}
  \UseMicrotypeSet[protrusion]{basicmath} % disable protrusion for tt fonts
}{}
\makeatletter
\@ifundefined{KOMAClassName}{% if non-KOMA class
  \IfFileExists{parskip.sty}{%
    \usepackage{parskip}
  }{% else
    \setlength{\parindent}{0pt}
    \setlength{\parskip}{6pt plus 2pt minus 1pt}}
}{% if KOMA class
  \KOMAoptions{parskip=half}}
\makeatother
\usepackage{xcolor}
\IfFileExists{xurl.sty}{\usepackage{xurl}}{} % add URL line breaks if available
\IfFileExists{bookmark.sty}{\usepackage{bookmark}}{\usepackage{hyperref}}
\hypersetup{
  pdftitle={Applied Microeconometrics - Assignment 1},
  pdfauthor={Walter Verwer \& Bas Machielsen},
  hidelinks,
  pdfcreator={LaTeX via pandoc}}
\urlstyle{same} % disable monospaced font for URLs
\usepackage[margin=1in]{geometry}
\usepackage{color}
\usepackage{fancyvrb}
\newcommand{\VerbBar}{|}
\newcommand{\VERB}{\Verb[commandchars=\\\{\}]}
\DefineVerbatimEnvironment{Highlighting}{Verbatim}{commandchars=\\\{\}}
% Add ',fontsize=\small' for more characters per line
\usepackage{framed}
\definecolor{shadecolor}{RGB}{248,248,248}
\newenvironment{Shaded}{\begin{snugshade}}{\end{snugshade}}
\newcommand{\AlertTok}[1]{\textcolor[rgb]{0.94,0.16,0.16}{#1}}
\newcommand{\AnnotationTok}[1]{\textcolor[rgb]{0.56,0.35,0.01}{\textbf{\textit{#1}}}}
\newcommand{\AttributeTok}[1]{\textcolor[rgb]{0.77,0.63,0.00}{#1}}
\newcommand{\BaseNTok}[1]{\textcolor[rgb]{0.00,0.00,0.81}{#1}}
\newcommand{\BuiltInTok}[1]{#1}
\newcommand{\CharTok}[1]{\textcolor[rgb]{0.31,0.60,0.02}{#1}}
\newcommand{\CommentTok}[1]{\textcolor[rgb]{0.56,0.35,0.01}{\textit{#1}}}
\newcommand{\CommentVarTok}[1]{\textcolor[rgb]{0.56,0.35,0.01}{\textbf{\textit{#1}}}}
\newcommand{\ConstantTok}[1]{\textcolor[rgb]{0.00,0.00,0.00}{#1}}
\newcommand{\ControlFlowTok}[1]{\textcolor[rgb]{0.13,0.29,0.53}{\textbf{#1}}}
\newcommand{\DataTypeTok}[1]{\textcolor[rgb]{0.13,0.29,0.53}{#1}}
\newcommand{\DecValTok}[1]{\textcolor[rgb]{0.00,0.00,0.81}{#1}}
\newcommand{\DocumentationTok}[1]{\textcolor[rgb]{0.56,0.35,0.01}{\textbf{\textit{#1}}}}
\newcommand{\ErrorTok}[1]{\textcolor[rgb]{0.64,0.00,0.00}{\textbf{#1}}}
\newcommand{\ExtensionTok}[1]{#1}
\newcommand{\FloatTok}[1]{\textcolor[rgb]{0.00,0.00,0.81}{#1}}
\newcommand{\FunctionTok}[1]{\textcolor[rgb]{0.00,0.00,0.00}{#1}}
\newcommand{\ImportTok}[1]{#1}
\newcommand{\InformationTok}[1]{\textcolor[rgb]{0.56,0.35,0.01}{\textbf{\textit{#1}}}}
\newcommand{\KeywordTok}[1]{\textcolor[rgb]{0.13,0.29,0.53}{\textbf{#1}}}
\newcommand{\NormalTok}[1]{#1}
\newcommand{\OperatorTok}[1]{\textcolor[rgb]{0.81,0.36,0.00}{\textbf{#1}}}
\newcommand{\OtherTok}[1]{\textcolor[rgb]{0.56,0.35,0.01}{#1}}
\newcommand{\PreprocessorTok}[1]{\textcolor[rgb]{0.56,0.35,0.01}{\textit{#1}}}
\newcommand{\RegionMarkerTok}[1]{#1}
\newcommand{\SpecialCharTok}[1]{\textcolor[rgb]{0.00,0.00,0.00}{#1}}
\newcommand{\SpecialStringTok}[1]{\textcolor[rgb]{0.31,0.60,0.02}{#1}}
\newcommand{\StringTok}[1]{\textcolor[rgb]{0.31,0.60,0.02}{#1}}
\newcommand{\VariableTok}[1]{\textcolor[rgb]{0.00,0.00,0.00}{#1}}
\newcommand{\VerbatimStringTok}[1]{\textcolor[rgb]{0.31,0.60,0.02}{#1}}
\newcommand{\WarningTok}[1]{\textcolor[rgb]{0.56,0.35,0.01}{\textbf{\textit{#1}}}}
\usepackage{graphicx}
\makeatletter
\def\maxwidth{\ifdim\Gin@nat@width>\linewidth\linewidth\else\Gin@nat@width\fi}
\def\maxheight{\ifdim\Gin@nat@height>\textheight\textheight\else\Gin@nat@height\fi}
\makeatother
% Scale images if necessary, so that they will not overflow the page
% margins by default, and it is still possible to overwrite the defaults
% using explicit options in \includegraphics[width, height, ...]{}
\setkeys{Gin}{width=\maxwidth,height=\maxheight,keepaspectratio}
% Set default figure placement to htbp
\makeatletter
\def\fps@figure{htbp}
\makeatother
\setlength{\emergencystretch}{3em} % prevent overfull lines
\providecommand{\tightlist}{%
  \setlength{\itemsep}{0pt}\setlength{\parskip}{0pt}}
\setcounter{secnumdepth}{-\maxdimen} % remove section numbering
\let\oldShaded\Shaded
\let\endoldShaded\endShaded
\renewenvironment{Shaded}{\footnotesize\oldShaded}{\endoldShaded}
\ifluatex
  \usepackage{selnolig}  % disable illegal ligatures
\fi

\title{Applied Microeconometrics - Assignment 1}
\author{Walter Verwer \& Bas Machielsen}
\date{8/31/2021}

\begin{document}
\maketitle

\begin{enumerate}
\def\labelenumi{\arabic{enumi}.}
\tightlist
\item
  Explain why first differencing the equation does not solve the
  endogeneity problem of lagged consumption.
\end{enumerate}

The first difference specification is:

\begin{align*}
(\log C_{it} - \log C_{it-1}) = \beta_1 \cdot (\log p_{it} - \log p_{it-1}) + \beta_2 \cdot (\log inc_{it} - \log inc_{it-1}) + \\
\beta_3 \cdot (\log ilop_{it} - \log ilop_{it-1}) + \beta_4 + \beta_5 \cdot (\log C_{it-1} - \log C_{it-2}) + u_{it} - u_{it-1}
\end{align*}

First difference estimation is just OLS estimation with transformed
data. For the OLS estimator (in general) to be consistent and unbiased,
we need \(\text{Cov}(X, U)=0\), where \(X\) is the matrix containing all
regressors. In the context of our transformed data, we need
\(\text{Cov}(\Delta X, \Delta U)=0\). One of the variables in
\(\Delta X\) is \(\Delta \log C_{it-1}\). If we evaluate the covariance
between \(\Delta \log C_{it-1}\) and \(\Delta U_{it}\), we find that:

\begin{align*}
&\text{Cov}(\Delta \log C_{it-1}, \Delta U_{it}) = \\
&\text{Cov}(\beta \Delta X_{it-1} + \beta_5 \Delta \log C_{it-2} + \Delta u_{it-1}, \Delta u_{it}) = \\
&\text{Cov}(\Delta U_{it-1}, \Delta U_{it}) \neq 0
\end{align*}

Since we observe that the exogeneity assumption is violated, we can
conclude that first differencing the equation does not solve the
endogeneity problem of lagged consumption.

\begin{enumerate}
\def\labelenumi{\arabic{enumi}.}
\setcounter{enumi}{1}
\tightlist
\item
  Anderson \& Hsiao propose a specific instrumental variable procedure
  for the model. Write down and perform the associated first stage
  regression. Comment on its outcomes.
\end{enumerate}

We have to keep in mind that the first-stage regression contains all the
exogenous regressors \(X\) from the second stage regression, plus the
instrument, \(C_{it-2}\). Hence, the first-stage model is:

\begin{align*}
\widehat{\log C_{it-1} - \log C_{it-2}} = \beta_1 \cdot (\log p_{it} - \log p_{it-1}) + \beta_2 \cdot (\log inc_{it} - \log inc_{it-1}) + \\
\beta_3 \cdot (\log ilop_{it} - \log ilop_{it-1}) + \beta_4 + \beta_5 \cdot (\log C_{it-2}) + u_{it-1} - u_{it-2}
\end{align*}

And the predicted values are to be used as follows in the second-stage
regression:

\begin{align*}
(\log C_{it} - \log C_{it-1}) = \beta_1 \cdot (\log p_{it} - \log p_{it-1}) + \beta_2 \cdot (\log inc_{it} - \log inc_{it-1}) + \\
\beta_3 \cdot (\log ilop_{it} - \log ilop_{it-1}) + \beta_4 + \beta_5 \cdot \widehat{(C_{it-1} - C_{it-2})} + u_{it} - u_{it-1}
\end{align*}

Using the data, we find the following first-stage regression (table
\ref{tab:reg}):

\begin{Shaded}
\begin{Highlighting}[]
\DocumentationTok{\#\# Create the first and second differences}
\NormalTok{dataset }\OtherTok{\textless{}{-}}\NormalTok{ dataset }\SpecialCharTok{\%\textgreater{}\%}
    \FunctionTok{group\_by}\NormalTok{(region) }\SpecialCharTok{\%\textgreater{}\%}
    \FunctionTok{mutate}\NormalTok{(}\FunctionTok{across}\NormalTok{(}\FunctionTok{contains}\NormalTok{(}\StringTok{"log"}\NormalTok{),}
                  \SpecialCharTok{\textasciitilde{}}\NormalTok{ .x }\SpecialCharTok{{-}}\NormalTok{ dplyr}\SpecialCharTok{::}\FunctionTok{lag}\NormalTok{(.x), }\AttributeTok{.names =} \StringTok{"l1\_\{.col\}"}\NormalTok{),}
           \FunctionTok{across}\NormalTok{(}\FunctionTok{starts\_with}\NormalTok{(}\StringTok{"log"}\NormalTok{),}
                  \SpecialCharTok{\textasciitilde{}}\NormalTok{ dplyr}\SpecialCharTok{::}\FunctionTok{lag}\NormalTok{(.x) }\SpecialCharTok{{-}}\NormalTok{ dplyr}\SpecialCharTok{::}\FunctionTok{lag}\NormalTok{(.x, }\DecValTok{2}\NormalTok{), }\AttributeTok{.names =} \StringTok{"l2\_\{.col\}"}\NormalTok{),}
           \AttributeTok{level\_quantity =}\NormalTok{ dplyr}\SpecialCharTok{::}\FunctionTok{lag}\NormalTok{(logquantity, }\DecValTok{2}\NormalTok{))}

\DocumentationTok{\#\# Run the first{-}stage regression}
\NormalTok{first\_stage\_reg }\OtherTok{\textless{}{-}} \FunctionTok{lm}\NormalTok{(}\AttributeTok{formula =} \StringTok{"l2\_logquantity \textasciitilde{} l1\_logprice + l1\_logincome + l1\_logillegal +}
\StringTok{   level\_quantity"}\NormalTok{,}
   \AttributeTok{data =}\NormalTok{ dataset)}
\end{Highlighting}
\end{Shaded}

Whereas the F-statistic is acceptable (higher than 10), it is not
\emph{much} higher than 10, leaving questions about the relevance of the
instrument. Indeed, the instrument seems to be lacking statistical
relevance, and thus predictive power. The coefficient on level quantity
is only -0.0150557 and insignificant at the 10\% level. This means that
consumption is \(C_{it-2}\) does not predict differences
\(C_{it-1} - C_{it-2}\) well, meaning there is no clear relationship
between absolute consumption and (near-)future increases/decreases of
consumption.

\begin{enumerate}
\def\labelenumi{\arabic{enumi}.}
\setcounter{enumi}{2}
\tightlist
\item
  Estimate the specification above using the Anderson \& Hsiao approach.
  Comment on the underlying assumptions, tabulate the results and
  comment on the outcomes.
\end{enumerate}

\begin{Shaded}
\begin{Highlighting}[]
\CommentTok{\# Use a package to estimate Anderson{-}Hsiao}
\NormalTok{dataset2 }\OtherTok{\textless{}{-}}\NormalTok{ plm}\SpecialCharTok{::}\FunctionTok{pdata.frame}\NormalTok{(dataset, }\FunctionTok{c}\NormalTok{(}\StringTok{"region"}\NormalTok{, }\StringTok{"year"}\NormalTok{))}

\NormalTok{anderson\_hsiao }\OtherTok{\textless{}{-}} \FunctionTok{plm}\NormalTok{(l1\_logquantity }\SpecialCharTok{\textasciitilde{}}\NormalTok{ l1\_logprice }\SpecialCharTok{+}\NormalTok{ l1\_logincome }\SpecialCharTok{+} 
\NormalTok{                          l1\_logillegal }\SpecialCharTok{+}\NormalTok{ l2\_logquantity }\SpecialCharTok{|} 
\NormalTok{                          l1\_logprice }\SpecialCharTok{+}\NormalTok{ l1\_logincome }\SpecialCharTok{+}\NormalTok{ l1\_logillegal }\SpecialCharTok{+} 
\NormalTok{                          level\_quantity,}
                      \AttributeTok{data=}\NormalTok{dataset2,}
                      \AttributeTok{model=}\StringTok{"pooling"}
\NormalTok{          )}

\CommentTok{\# Compare with Manual 2SLS}
\NormalTok{dataset }\OtherTok{\textless{}{-}}\NormalTok{ modelr}\SpecialCharTok{::}\FunctionTok{add\_predictions}\NormalTok{(dataset, first\_stage\_reg) }\SpecialCharTok{\%\textgreater{}\%}
       \FunctionTok{rename}\NormalTok{(}\StringTok{"c\_instrumented"}\OtherTok{=}\NormalTok{pred) }

\NormalTok{manual\_2sls }\OtherTok{\textless{}{-}} \FunctionTok{lm}\NormalTok{(}\AttributeTok{data=}\NormalTok{dataset, }
   \AttributeTok{formula =}\NormalTok{ l1\_logquantity }\SpecialCharTok{\textasciitilde{}}\NormalTok{ l1\_logprice }\SpecialCharTok{+}\NormalTok{ l1\_logincome }\SpecialCharTok{+}\NormalTok{ l1\_logillegal }\SpecialCharTok{+}\NormalTok{ c\_instrumented) }

\FunctionTok{stargazer}\NormalTok{(first\_stage\_reg, anderson\_hsiao, manual\_2sls, }
          \AttributeTok{label =} \StringTok{"tab:reg"}\NormalTok{, }\AttributeTok{header=}\ConstantTok{FALSE}\NormalTok{, }\AttributeTok{model.names =} \ConstantTok{FALSE}\NormalTok{,}
          \AttributeTok{column.sep.width=}\StringTok{"{-}5pt"}\NormalTok{,}
          \AttributeTok{dep.var.labels=}\FunctionTok{c}\NormalTok{(}\StringTok{"$C\_\{it{-}1\} {-} C\_\{it{-}2\}$"}\NormalTok{, }
                           \StringTok{"$C\_\{it\} {-} C\_\{it{-}1\}$"}\NormalTok{,}
                           \StringTok{"$C\_\{it\} {-} C\_\{it{-}1\}$"}\NormalTok{),}
          \AttributeTok{column.labels =} \FunctionTok{c}\NormalTok{(}\StringTok{"First{-}Stage"}\NormalTok{, }\StringTok{"A{-}H"}\NormalTok{, }\StringTok{"Manual 2SLS"}\NormalTok{),}
              \AttributeTok{omit.stat =} \FunctionTok{c}\NormalTok{(}\StringTok{"ll"}\NormalTok{, }\StringTok{"ser"}\NormalTok{, }\StringTok{"rsq"}\NormalTok{))}
\end{Highlighting}
\end{Shaded}

\begin{table}[!htbp] \centering 
  \caption{} 
  \label{tab:reg} 
\begin{tabular}{@{\extracolsep{-5pt}}lccc} 
\\[-1.8ex]\hline 
\hline \\[-1.8ex] 
 & \multicolumn{3}{c}{\textit{Dependent variable:}} \\ 
\cline{2-4} 
\\[-1.8ex] & $C_{it-1} - C_{it-2}$ & \multicolumn{2}{c}{$C_{it} - C_{it-1}$} \\ 
 & First-Stage & A-H & Manual 2SLS \\ 
\\[-1.8ex] & (1) & (2) & (3)\\ 
\hline \\[-1.8ex] 
 l1\_logprice & $-$0.617$^{***}$ & 0.022 & 0.022 \\ 
  & (0.089) & (0.556) & (0.338) \\ 
  & & & \\ 
 l1\_logincome & $-$0.836$^{***}$ & 1.878$^{**}$ & 1.878$^{***}$ \\ 
  & (0.219) & (0.762) & (0.463) \\ 
  & & & \\ 
 l1\_logillegal & $-$0.005 & $-$0.029 & $-$0.029$^{**}$ \\ 
  & (0.013) & (0.018) & (0.011) \\ 
  & & & \\ 
 level\_quantity & $-$0.015 &  &  \\ 
  & (0.009) &  &  \\ 
  & & & \\ 
 l2\_logquantity &  & 1.470$^{*}$ &  \\ 
  &  & (0.851) &  \\ 
  & & & \\ 
 c\_instrumented &  &  & 1.470$^{***}$ \\ 
  &  &  & (0.517) \\ 
  & & & \\ 
 Constant & 0.086 & $-$0.002 & $-$0.002 \\ 
  & (0.061) & (0.023) & (0.014) \\ 
  & & & \\ 
\hline \\[-1.8ex] 
Observations & 308 & 308 & 308 \\ 
Adjusted R$^{2}$ & 0.157 & 0.275 & 0.398 \\ 
F Statistic (df = 4; 303) & 15.264$^{***}$ & 76.600$^{***}$ & 51.812$^{***}$ \\ 
\hline 
\hline \\[-1.8ex] 
\textit{Note:}  & \multicolumn{3}{r}{$^{*}$p$<$0.1; $^{**}$p$<$0.05; $^{***}$p$<$0.01} \\ 
\end{tabular} 
\end{table}

The table is displayed below. The estimates from models (2) and (3) in
tabel \ref{tab:reg} are the same. Only the variance of the
2SLS-estimator is off.

\begin{enumerate}
\def\labelenumi{\arabic{enumi}.}
\setcounter{enumi}{3}
\tightlist
\item
  Describe the Arellano \& Bond GMM estimator for this model.
\end{enumerate}

In general the Arellano \& Bond GMM estimator aims to use lagged values
of endogenous regressors as an instrument. All possible moment
conditions are for \(t=2,\dots,T\) and \(k=2,\dots,t\), applied to the
current model given by:

\begin{equation*}
\mathbb{E}[\log(C_{it-k})(u_{it}-u_{it-1})]
\end{equation*}

\begin{enumerate}
\def\labelenumi{\arabic{enumi}.}
\setcounter{enumi}{4}
\tightlist
\item
  Estimate the model parameters using the Arellano \& Bond estimator,
  tabulate the results and discuss the parameter estimates.
\end{enumerate}

\begin{Shaded}
\begin{Highlighting}[]
\NormalTok{plm}\SpecialCharTok{::}\NormalTok{pgmm}
\end{Highlighting}
\end{Shaded}

function (formula, data, subset, na.action, effect = c(``twoways'',
``individual''), model = c(``onestep'', ``twosteps''), collapse = FALSE,
lost.ts = NULL, transformation = c(``d'', ``ld''), fsm = NULL, index =
NULL, \ldots) \{ cl \textless- match.call(expand.dots = TRUE) effect
\textless- match.arg(effect) model \textless- match.arg(model)
transformation \textless- match.arg(transformation) namesV \textless-
NULL if (inherits(formula, ``dynformula'') \textbar\textbar{}
length(Formula(formula)){[}2L{]} == 1) \{ if (!inherits(formula,
``dynformula'')) \{ formula \textless- match.call(expand.dots = TRUE) m
\textless- match(c(``formula'', ``lag.form'', ``diff.form'',
``log.form''), names(formula), 0) formula \textless- formula{[}c(1,
m){]} formula{[}{[}1{]}{]} \textless- as.name(``dynformula'') formula
\textless-
cl\(formula <- eval(formula, parent.frame())  }  response.name <- paste(deparse(formula[[2L]]))  main.lags <- attr(formula, "lag")  if (length(main.lags[[1L]]) == 1 && main.lags[[1L]] >  1)  main.lags[[1L]] <- c(1, main.lags[[1L]])  main.lags[2:length(main.lags)] <- lapply(main.lags[2:length(main.lags)],  function(x) {  if (length(x) == 1 && x != 0)  x <- c(0, x)  x  })  main.form <- dynterms2formula(main.lags, response.name)  dots <- list(...)  gmm.inst <- dots\)gmm.inst
lag.gmm \textless- dots\(lag.gmm  instruments <- dots\)instruments
gmm.form \textless- dynformula(gmm.inst, lag.form = lag.gmm) gmm.lags
\textless- attr(gmm.form, ``lag'') gmm.lags \textless- lapply(gmm.lags,
function(x) min(x):max(x)) gmm.form \textless-
dynterms2formula(gmm.lags) formula \textless- as.Formula(main.form,
gmm.form) \} x \textless- formula if (!inherits(x, ``Formula'')) x
\textless- Formula(formula) gmm.form \textless- formula(x, rhs = 2, lhs
= 0) gmm.lags \textless- dynterms(gmm.form) cardW \textless-
length(gmm.lags) if (is.null(names(collapse))) \{ if (length(collapse)
== 1) \{ collapse \textless- as.vector(rep(collapse, cardW), mode =
``list'') \} else \{ if (length(collapse) != cardW) stop(``the collapse
vector has a wrong length'') \} names(collapse) \textless-
names(gmm.lags) \} else \{ if (any(!(names(collapse) \%in\%
names(gmm.lags)))) stop(``unknown names in the collapse vector'') else
\{ bcollapse \textless- as.vector(rep(FALSE, cardW), mode = ``list'')
names(bcollapse) \textless- names(gmm.lags)
bcollapse{[}names(collapse){]} \textless- collapse collapse \textless-
bcollapse \} \} main.form \textless- formula(x, rhs = 1, lhs = 1)
main.lags \textless- dynterms(main.form) if (length(x){[}2L{]} == 3) \{
normal.instruments \textless- TRUE inst.form \textless- formula(x, rhs =
3, lhs = 0) inst.form \textless- update(main.form, inst.form) inst.form
\textless- formula(Formula(inst.form), lhs = 0) inst.lags \textless-
dynterms(inst.form) \} else \{ iv \textless-
names(main.lags){[}!names(main.lags) \%in\% names(gmm.lags){]} inst.lags
\textless- main.lags{[}iv{]} if (length(inst.lags) \textgreater{} 0) \{
normal.instruments \textless- TRUE inst.form \textless-
dynterms2formula(inst.lags) \} else \{ normal.instruments \textless-
FALSE inst.form \textless- NULL inst.lags \textless- NULL \} \} if
(!is.null(lost.ts)) \{ if (!is.numeric(lost.ts)) stop(``lost.ts should
be numeric'') lost.ts \textless- as.numeric(lost.ts) if
(!(length(lost.ts) \%in\% c(1, 2))) stop(``lost.ts should be of length 1
or 2'') TL1 \textless- lost.ts{[}1L{]} TL2 \textless-
ifelse(length(lost.ts) == 1, TL1 - 1, lost.ts{[}2L{]}) \} else \{
gmm.minlag \textless- min(sapply(gmm.lags, min)) if
(!is.null(inst.lags)) inst.maxlag \textless- max(sapply(inst.lags, max))
else inst.maxlag \textless- 0 main.maxlag \textless-
max(sapply(main.lags, max)) TL1 \textless- max(main.maxlag + 1,
inst.maxlag + 1, gmm.minlag) TL2 \textless- max(main.maxlag,
inst.maxlag, gmm.minlag - 1) TL1 \textless- max(main.maxlag + 1,
gmm.minlag) TL2 \textless- max(main.maxlag, gmm.minlag - 1) \} gmm.form
\textless- as.formula(paste(``\textasciitilde{}'',
paste(names(gmm.lags), collapse = ``+''))) if (!is.null(inst.form)) Form
\textless- as.Formula(main.form, gmm.form, inst.form) else Form
\textless- as.Formula(main.form, gmm.form) mf \textless-
match.call(expand.dots = FALSE) m \textless- match(c(``formula'',
``data'', ``subset'', ``na.action'', ``index''), names(mf), 0) mf
\textless- mf{[}c(1, m){]}
mf\(drop.unused.levels <- TRUE  mf[[1L]] <- as.name("plm")  mf\)model
\textless- NA mf\(formula <- Form  mf\)na.action \textless- ``na.pass''
mf\(subset <- NULL  data <- eval(mf, parent.frame())  index <- index(data)  N <- length(levels(index[[1L]]))  T <- length(levels(index[[2L]]))  pdim <- pdim(data)  balanced <- pdim\)balanced
if (!balanced) \{ un.id \textless- sort(unique(index(data, ``id'')))
un.time \textless- sort(unique(index(data, ``time''))) rownames(data)
\textless- paste(index(data, ``id''), index(data, ``time''), sep =
``.'') allRows \textless- as.character(t(outer(un.id, un.time, paste,
sep = ``.''))) data \textless- data{[}allRows, {]} rownames(data)
\textless- allRows index \textless- data.frame(id = rep(un.id, each =
length(un.time)), time = rep(un.time, length(un.id)), row.names =
rownames(data)) class(index) \textless- c(``pindex'', ``data.frame'')
attr(data, ``index'') \textless- index \} attr(data, ``formula'')
\textless- formula(main.form) yX \textless- extract.data(data)
names.coef \textless- colnames(yX{[}{[}1L{]}{]}){[}-1L{]} if
(normal.instruments) \{ attr(data, ``formula'') \textless- inst.form Z
\textless- extract.data(data) \} else Z \textless- NULL attr(data,
``formula'') \textless- gmm.form W \textless- extract.data(data,
as.matrix = FALSE) W1 \textless- lapply(W, function(x) \{ u \textless-
mapply(makegmm, x, gmm.lags, TL1, collapse, SIMPLIFY = FALSE) u
\textless- matrix(unlist(u), nrow = nrow(u{[}{[}1L{]}{]})) u \}) yX1
\textless- lapply(yX, function(x) \{ xd \textless- diff(x) xd \textless-
xd{[}-c(1:(TL1 - 1)), , drop = FALSE{]} xd \}) if (normal.instruments)
\{ Z1 \textless- lapply(Z, function(x) \{ xd \textless- diff(x) xd
\textless- xd{[}-c(1:(TL1 - 1)), , drop = FALSE{]} xd \}) \} if
(transformation == ``ld'') \{ W2 \textless- lapply(W, function(x) \{ u
\textless- mapply(makeW2, x, collapse, SIMPLIFY = FALSE) nrow.ud
\textless- ifelse(TL2 == 1, T - 2, T - TL2) u \textless-
matrix(unlist(u), nrow = nrow.ud) if (TL2 == 1) u \textless- rbind(0, u)
u \}) yX2 \textless- lapply(yX, function(x) \{ x \textless-
x{[}-c(0:TL2), , drop = FALSE{]} x \}) if (normal.instruments) \{ Z2
\textless- lapply(Z, function(x) \{ x \textless- x{[}-c(0:TL2), , drop =
FALSE{]} x \}) \} \} if (effect == ``twoways'') \{ namesV \textless-
levels(index(data, which = ``time'')) if (transformation == ``d'') \{ V1
\textless- td.model.diff \textless- diff(diag(1, T - TL1 + 1)){[}, -1{]}
namesV \textless- namesV{[}-c(0:(TL1)){]} \} else \{ td \textless-
cbind(1, rbind(0, diag(1, T - 1))) V2 \textless- td{[}-c(1:TL2), -c(2:(2
+ TL2 - 1)){]} V1 \textless- diff(V2) namesV \textless-
c(``(intercept)'', namesV{[}-c(0:TL2 + 1){]}) \} for (i in 1:N) \{
yX1{[}{[}i{]}{]} \textless- cbind(yX1{[}{[}i{]}{]}, V1) if
(transformation == ``d'') \{ W1{[}{[}i{]}{]} \textless-
cbind(W1{[}{[}i{]}{]}, V1) \} else \{ W2{[}{[}i{]}{]} \textless-
cbind(W2{[}{[}i{]}{]}, V2) yX2{[}{[}i{]}{]} \textless-
cbind(yX2{[}{[}i{]}{]}, V2) \} \} \} if (effect == ``individual'' \&\&
transformation == ``ld'') \{ namesV \textless- levels(index(data, which
= ``time'')) namesV \textless- c(``(intercept)'', namesV{[}-c(0:TL2 +
1){]}) \} for (i in 1:N) \{ narows \textless- apply(yX1{[}{[}i{]}{]}, 1,
function(z) anyNA(z)) yX1{[}{[}i{]}{]}{[}narows, {]} \textless- 0
W1{[}{[}i{]}{]}{[}is.na(W1{[}{[}i{]}{]}){]} \textless- 0
W1{[}{[}i{]}{]}{[}narows, {]} \textless- 0 if (normal.instruments) \{
Z1{[}{[}i{]}{]}{[}is.na(Z1{[}{[}i{]}{]}){]} \textless- 0
Z1{[}{[}i{]}{]}{[}narows, {]} \textless- 0 \} if (transformation ==
``ld'') \{ narows \textless- apply(yX2{[}{[}i{]}{]}, 1, function(z)
anyNA(z)) yX2{[}{[}i{]}{]}{[}narows, {]} \textless- 0
W2{[}{[}i{]}{]}{[}is.na(W2{[}{[}i{]}{]}){]} \textless- 0
W2{[}{[}i{]}{]}{[}narows, {]} \textless- 0 if (normal.instruments) \{
Z2{[}{[}i{]}{]}{[}is.na(Z2{[}{[}i{]}{]}){]} \textless- 0
Z2{[}{[}i{]}{]}{[}narows, {]} \textless- 0 \} \} \} if (transformation
== ``ld'') \{ for (i in 1:N) \{ W1{[}{[}i{]}{]} \textless-
bdiag(W1{[}{[}i{]}{]}, W2{[}{[}i{]}{]}) yX1{[}{[}i{]}{]} \textless-
rbind(yX1{[}{[}i{]}{]}, yX2{[}{[}i{]}{]}) if (normal.instruments)
Z1{[}{[}i{]}{]} \textless- bdiag(Z1{[}{[}i{]}{]}, Z2{[}{[}i{]}{]}) \} \}
if (normal.instruments) \{ for (i in 1:N) W1{[}{[}i{]}{]} \textless-
cbind(W1{[}{[}i{]}{]}, Z1{[}{[}i{]}{]}) \} W \textless- W1 yX \textless-
yX1 if (transformation == ``d'') A1 \textless- tcrossprod(diff(diag(1, T
- TL1 + 1))) if (transformation == ``ld'') A1 \textless- FSM(T - TL2,
``full'') WX \textless- mapply(function(x, y) crossprod(x, y), W, yX,
SIMPLIFY = FALSE) Wy \textless- lapply(WX, function(x) x{[}, 1{]}) WX
\textless- lapply(WX, function(x) x{[}, -1, drop = FALSE{]}) A1
\textless- lapply(W, function(x) crossprod(t(crossprod(x, A1)), x)) A1
\textless- Reduce(``+'', A1) minevA1 \textless-
min(eigen(A1)\(values)  eps <- 1e-09  if (minevA1 < eps) {  A1 <- ginv(A1) * length(W)  warning("the first-step matrix is singular, a general inverse is used")  }  else A1 <- solve(A1) * length(W)  WX <- Reduce("+", WX)  Wy <- Reduce("+", Wy)  B1 <- solve(crossprod(WX, t(crossprod(WX, A1))))  Y1 <- crossprod(t(crossprod(WX, A1)), Wy)  coefficients <- as.numeric(crossprod(B1, Y1))  if (effect == "twoways")  names.coef <- c(names.coef, namesV)  names(coefficients) <- names.coef  residuals <- lapply(yX, function(x) as.vector(x[, 1] - crossprod(t(x[,  -1, drop = FALSE]), coefficients)))  outresid <- lapply(residuals, function(x) outer(x, x))  A2 <- mapply(function(x, y) crossprod(t(crossprod(x, y)),  x), W, outresid, SIMPLIFY = FALSE)  A2 <- Reduce("+", A2)  minevA2 <- min(eigen(A2)\)values)
eps \textless- 1e-09 if (minevA2 \textless{} eps) \{ A2 \textless-
ginv(A2) warning(``the second-step matrix is singular, a general inverse
is used'') \} else A2 \textless- solve(A2) B2 \textless-
solve(crossprod(WX, t(crossprod(WX, A2)))) if (model == ``twosteps'') \{
coef1s \textless- coefficients Y2 \textless- crossprod(t(crossprod(WX,
A2)), Wy) coefficients \textless- as.numeric(crossprod(B2, Y2))
names(coefficients) \textless- names.coef vcov \textless- B2 \} else
vcov \textless- B1 rownames(vcov) \textless- colnames(vcov) \textless-
names.coef residuals \textless- lapply(yX, function(x) \{ nz \textless-
rownames(x) z \textless- as.vector(x{[}, 1{]} - crossprod(t(x{[}, -1,
drop = FALSE{]}), coefficients)) names(z) \textless- nz z \})
fitted.values \textless- mapply(function(x, y) x{[}, 1{]} - y, yX,
residuals) if (model == ``twosteps'') coefficients \textless-
list(coef1s, coefficients) args \textless- list(model = model, effect =
effect, transformation = transformation, namest = namesV) result
\textless- list(coefficients = coefficients, residuals = residuals, vcov
= vcov, fitted.values = fitted.values, model = yX, W = W, A1 = A1, A2 =
A2, call = cl, args = args) result \textless- structure(result, class =
c(``pgmm'', ``panelmodel''), pdim = pdim) result \} \textless bytecode:
0x0000000019945dd0\textgreater{} \textless environment:
namespace:plm\textgreater{}

\begin{Shaded}
\begin{Highlighting}[]
\CommentTok{\# Here, we use the \_normal\_ specification as default and not the first difference:}
\CommentTok{\# The package will transform the data accordingly}
\NormalTok{arellano\_bond }\OtherTok{\textless{}{-}} \FunctionTok{pgmm}\NormalTok{(}\AttributeTok{data =}\NormalTok{ dataset2,}
\NormalTok{                          logquantity }\SpecialCharTok{\textasciitilde{}}\NormalTok{ logincome }\SpecialCharTok{+}\NormalTok{ logprice }\SpecialCharTok{+} 
\NormalTok{                            logillegal }\SpecialCharTok{+} \FunctionTok{lag}\NormalTok{(logquantity, }\DecValTok{1}\NormalTok{)}
                          \SpecialCharTok{|} \FunctionTok{lag}\NormalTok{(logquantity, }\DecValTok{2}\SpecialCharTok{:}\DecValTok{99}\NormalTok{), }\AttributeTok{transformation =} \StringTok{\textquotesingle{}d\textquotesingle{}}\NormalTok{)}

\FunctionTok{stargazer}\NormalTok{(arellano\_bond, }\AttributeTok{label =} \StringTok{"tab:reg\_ab"}\NormalTok{, }\AttributeTok{header=}\ConstantTok{FALSE}\NormalTok{, }\AttributeTok{model.names =} \ConstantTok{FALSE}\NormalTok{,}
          \AttributeTok{column.sep.width=}\StringTok{"{-}5pt"}\NormalTok{,}
          \AttributeTok{column.labels =} \FunctionTok{c}\NormalTok{(}\StringTok{\textquotesingle{}Arellano \& Bond\textquotesingle{}}\NormalTok{),}
              \AttributeTok{omit.stat =} \FunctionTok{c}\NormalTok{(}\StringTok{"ll"}\NormalTok{, }\StringTok{"ser"}\NormalTok{, }\StringTok{"rsq"}\NormalTok{))}
\end{Highlighting}
\end{Shaded}

\begin{table}[!htbp] \centering 
  \caption{} 
  \label{tab:reg_ab} 
\begin{tabular}{@{\extracolsep{-5pt}}lc} 
\\[-1.8ex]\hline 
\hline \\[-1.8ex] 
 & \multicolumn{1}{c}{\textit{Dependent variable:}} \\ 
\cline{2-2} 
\\[-1.8ex] & logquantity \\ 
 & Arellano & Bond \\ 
\hline \\[-1.8ex] 
 logincome & 0.576$^{**}$ \\ 
  & (0.248) \\ 
  & \\ 
 logprice & $-$0.410$^{**}$ \\ 
  & (0.190) \\ 
  & \\ 
 logillegal & $-$0.041$^{***}$ \\ 
  & (0.013) \\ 
  & \\ 
 lag(logquantity, 1) & 0.828$^{***}$ \\ 
  & (0.039) \\ 
  & \\ 
\hline \\[-1.8ex] 
Observations & 22 \\ 
\hline 
\hline \\[-1.8ex] 
\textit{Note:}  & \multicolumn{1}{r}{$^{*}$p$<$0.1; $^{**}$p$<$0.05; $^{***}$p$<$0.01} \\ 
\end{tabular} 
\end{table}

\begin{enumerate}
\def\labelenumi{\arabic{enumi}.}
\setcounter{enumi}{5}
\item
  What is in your estimate for the short-run and the long-run price
  elasticity of opium?
\item
  Now estimate the model parameters using the system estimator (Blundell
  \& Bond). Tabulate results, compute the elasticities (as in 6.).
\end{enumerate}

\begin{Shaded}
\begin{Highlighting}[]
\NormalTok{plm}\SpecialCharTok{::}\NormalTok{pgmm}
\end{Highlighting}
\end{Shaded}

function (formula, data, subset, na.action, effect = c(``twoways'',
``individual''), model = c(``onestep'', ``twosteps''), collapse = FALSE,
lost.ts = NULL, transformation = c(``d'', ``ld''), fsm = NULL, index =
NULL, \ldots) \{ cl \textless- match.call(expand.dots = TRUE) effect
\textless- match.arg(effect) model \textless- match.arg(model)
transformation \textless- match.arg(transformation) namesV \textless-
NULL if (inherits(formula, ``dynformula'') \textbar\textbar{}
length(Formula(formula)){[}2L{]} == 1) \{ if (!inherits(formula,
``dynformula'')) \{ formula \textless- match.call(expand.dots = TRUE) m
\textless- match(c(``formula'', ``lag.form'', ``diff.form'',
``log.form''), names(formula), 0) formula \textless- formula{[}c(1,
m){]} formula{[}{[}1{]}{]} \textless- as.name(``dynformula'') formula
\textless-
cl\(formula <- eval(formula, parent.frame())  }  response.name <- paste(deparse(formula[[2L]]))  main.lags <- attr(formula, "lag")  if (length(main.lags[[1L]]) == 1 && main.lags[[1L]] >  1)  main.lags[[1L]] <- c(1, main.lags[[1L]])  main.lags[2:length(main.lags)] <- lapply(main.lags[2:length(main.lags)],  function(x) {  if (length(x) == 1 && x != 0)  x <- c(0, x)  x  })  main.form <- dynterms2formula(main.lags, response.name)  dots <- list(...)  gmm.inst <- dots\)gmm.inst
lag.gmm \textless- dots\(lag.gmm  instruments <- dots\)instruments
gmm.form \textless- dynformula(gmm.inst, lag.form = lag.gmm) gmm.lags
\textless- attr(gmm.form, ``lag'') gmm.lags \textless- lapply(gmm.lags,
function(x) min(x):max(x)) gmm.form \textless-
dynterms2formula(gmm.lags) formula \textless- as.Formula(main.form,
gmm.form) \} x \textless- formula if (!inherits(x, ``Formula'')) x
\textless- Formula(formula) gmm.form \textless- formula(x, rhs = 2, lhs
= 0) gmm.lags \textless- dynterms(gmm.form) cardW \textless-
length(gmm.lags) if (is.null(names(collapse))) \{ if (length(collapse)
== 1) \{ collapse \textless- as.vector(rep(collapse, cardW), mode =
``list'') \} else \{ if (length(collapse) != cardW) stop(``the collapse
vector has a wrong length'') \} names(collapse) \textless-
names(gmm.lags) \} else \{ if (any(!(names(collapse) \%in\%
names(gmm.lags)))) stop(``unknown names in the collapse vector'') else
\{ bcollapse \textless- as.vector(rep(FALSE, cardW), mode = ``list'')
names(bcollapse) \textless- names(gmm.lags)
bcollapse{[}names(collapse){]} \textless- collapse collapse \textless-
bcollapse \} \} main.form \textless- formula(x, rhs = 1, lhs = 1)
main.lags \textless- dynterms(main.form) if (length(x){[}2L{]} == 3) \{
normal.instruments \textless- TRUE inst.form \textless- formula(x, rhs =
3, lhs = 0) inst.form \textless- update(main.form, inst.form) inst.form
\textless- formula(Formula(inst.form), lhs = 0) inst.lags \textless-
dynterms(inst.form) \} else \{ iv \textless-
names(main.lags){[}!names(main.lags) \%in\% names(gmm.lags){]} inst.lags
\textless- main.lags{[}iv{]} if (length(inst.lags) \textgreater{} 0) \{
normal.instruments \textless- TRUE inst.form \textless-
dynterms2formula(inst.lags) \} else \{ normal.instruments \textless-
FALSE inst.form \textless- NULL inst.lags \textless- NULL \} \} if
(!is.null(lost.ts)) \{ if (!is.numeric(lost.ts)) stop(``lost.ts should
be numeric'') lost.ts \textless- as.numeric(lost.ts) if
(!(length(lost.ts) \%in\% c(1, 2))) stop(``lost.ts should be of length 1
or 2'') TL1 \textless- lost.ts{[}1L{]} TL2 \textless-
ifelse(length(lost.ts) == 1, TL1 - 1, lost.ts{[}2L{]}) \} else \{
gmm.minlag \textless- min(sapply(gmm.lags, min)) if
(!is.null(inst.lags)) inst.maxlag \textless- max(sapply(inst.lags, max))
else inst.maxlag \textless- 0 main.maxlag \textless-
max(sapply(main.lags, max)) TL1 \textless- max(main.maxlag + 1,
inst.maxlag + 1, gmm.minlag) TL2 \textless- max(main.maxlag,
inst.maxlag, gmm.minlag - 1) TL1 \textless- max(main.maxlag + 1,
gmm.minlag) TL2 \textless- max(main.maxlag, gmm.minlag - 1) \} gmm.form
\textless- as.formula(paste(``\textasciitilde{}'',
paste(names(gmm.lags), collapse = ``+''))) if (!is.null(inst.form)) Form
\textless- as.Formula(main.form, gmm.form, inst.form) else Form
\textless- as.Formula(main.form, gmm.form) mf \textless-
match.call(expand.dots = FALSE) m \textless- match(c(``formula'',
``data'', ``subset'', ``na.action'', ``index''), names(mf), 0) mf
\textless- mf{[}c(1, m){]}
mf\(drop.unused.levels <- TRUE  mf[[1L]] <- as.name("plm")  mf\)model
\textless- NA mf\(formula <- Form  mf\)na.action \textless- ``na.pass''
mf\(subset <- NULL  data <- eval(mf, parent.frame())  index <- index(data)  N <- length(levels(index[[1L]]))  T <- length(levels(index[[2L]]))  pdim <- pdim(data)  balanced <- pdim\)balanced
if (!balanced) \{ un.id \textless- sort(unique(index(data, ``id'')))
un.time \textless- sort(unique(index(data, ``time''))) rownames(data)
\textless- paste(index(data, ``id''), index(data, ``time''), sep =
``.'') allRows \textless- as.character(t(outer(un.id, un.time, paste,
sep = ``.''))) data \textless- data{[}allRows, {]} rownames(data)
\textless- allRows index \textless- data.frame(id = rep(un.id, each =
length(un.time)), time = rep(un.time, length(un.id)), row.names =
rownames(data)) class(index) \textless- c(``pindex'', ``data.frame'')
attr(data, ``index'') \textless- index \} attr(data, ``formula'')
\textless- formula(main.form) yX \textless- extract.data(data)
names.coef \textless- colnames(yX{[}{[}1L{]}{]}){[}-1L{]} if
(normal.instruments) \{ attr(data, ``formula'') \textless- inst.form Z
\textless- extract.data(data) \} else Z \textless- NULL attr(data,
``formula'') \textless- gmm.form W \textless- extract.data(data,
as.matrix = FALSE) W1 \textless- lapply(W, function(x) \{ u \textless-
mapply(makegmm, x, gmm.lags, TL1, collapse, SIMPLIFY = FALSE) u
\textless- matrix(unlist(u), nrow = nrow(u{[}{[}1L{]}{]})) u \}) yX1
\textless- lapply(yX, function(x) \{ xd \textless- diff(x) xd \textless-
xd{[}-c(1:(TL1 - 1)), , drop = FALSE{]} xd \}) if (normal.instruments)
\{ Z1 \textless- lapply(Z, function(x) \{ xd \textless- diff(x) xd
\textless- xd{[}-c(1:(TL1 - 1)), , drop = FALSE{]} xd \}) \} if
(transformation == ``ld'') \{ W2 \textless- lapply(W, function(x) \{ u
\textless- mapply(makeW2, x, collapse, SIMPLIFY = FALSE) nrow.ud
\textless- ifelse(TL2 == 1, T - 2, T - TL2) u \textless-
matrix(unlist(u), nrow = nrow.ud) if (TL2 == 1) u \textless- rbind(0, u)
u \}) yX2 \textless- lapply(yX, function(x) \{ x \textless-
x{[}-c(0:TL2), , drop = FALSE{]} x \}) if (normal.instruments) \{ Z2
\textless- lapply(Z, function(x) \{ x \textless- x{[}-c(0:TL2), , drop =
FALSE{]} x \}) \} \} if (effect == ``twoways'') \{ namesV \textless-
levels(index(data, which = ``time'')) if (transformation == ``d'') \{ V1
\textless- td.model.diff \textless- diff(diag(1, T - TL1 + 1)){[}, -1{]}
namesV \textless- namesV{[}-c(0:(TL1)){]} \} else \{ td \textless-
cbind(1, rbind(0, diag(1, T - 1))) V2 \textless- td{[}-c(1:TL2), -c(2:(2
+ TL2 - 1)){]} V1 \textless- diff(V2) namesV \textless-
c(``(intercept)'', namesV{[}-c(0:TL2 + 1){]}) \} for (i in 1:N) \{
yX1{[}{[}i{]}{]} \textless- cbind(yX1{[}{[}i{]}{]}, V1) if
(transformation == ``d'') \{ W1{[}{[}i{]}{]} \textless-
cbind(W1{[}{[}i{]}{]}, V1) \} else \{ W2{[}{[}i{]}{]} \textless-
cbind(W2{[}{[}i{]}{]}, V2) yX2{[}{[}i{]}{]} \textless-
cbind(yX2{[}{[}i{]}{]}, V2) \} \} \} if (effect == ``individual'' \&\&
transformation == ``ld'') \{ namesV \textless- levels(index(data, which
= ``time'')) namesV \textless- c(``(intercept)'', namesV{[}-c(0:TL2 +
1){]}) \} for (i in 1:N) \{ narows \textless- apply(yX1{[}{[}i{]}{]}, 1,
function(z) anyNA(z)) yX1{[}{[}i{]}{]}{[}narows, {]} \textless- 0
W1{[}{[}i{]}{]}{[}is.na(W1{[}{[}i{]}{]}){]} \textless- 0
W1{[}{[}i{]}{]}{[}narows, {]} \textless- 0 if (normal.instruments) \{
Z1{[}{[}i{]}{]}{[}is.na(Z1{[}{[}i{]}{]}){]} \textless- 0
Z1{[}{[}i{]}{]}{[}narows, {]} \textless- 0 \} if (transformation ==
``ld'') \{ narows \textless- apply(yX2{[}{[}i{]}{]}, 1, function(z)
anyNA(z)) yX2{[}{[}i{]}{]}{[}narows, {]} \textless- 0
W2{[}{[}i{]}{]}{[}is.na(W2{[}{[}i{]}{]}){]} \textless- 0
W2{[}{[}i{]}{]}{[}narows, {]} \textless- 0 if (normal.instruments) \{
Z2{[}{[}i{]}{]}{[}is.na(Z2{[}{[}i{]}{]}){]} \textless- 0
Z2{[}{[}i{]}{]}{[}narows, {]} \textless- 0 \} \} \} if (transformation
== ``ld'') \{ for (i in 1:N) \{ W1{[}{[}i{]}{]} \textless-
bdiag(W1{[}{[}i{]}{]}, W2{[}{[}i{]}{]}) yX1{[}{[}i{]}{]} \textless-
rbind(yX1{[}{[}i{]}{]}, yX2{[}{[}i{]}{]}) if (normal.instruments)
Z1{[}{[}i{]}{]} \textless- bdiag(Z1{[}{[}i{]}{]}, Z2{[}{[}i{]}{]}) \} \}
if (normal.instruments) \{ for (i in 1:N) W1{[}{[}i{]}{]} \textless-
cbind(W1{[}{[}i{]}{]}, Z1{[}{[}i{]}{]}) \} W \textless- W1 yX \textless-
yX1 if (transformation == ``d'') A1 \textless- tcrossprod(diff(diag(1, T
- TL1 + 1))) if (transformation == ``ld'') A1 \textless- FSM(T - TL2,
``full'') WX \textless- mapply(function(x, y) crossprod(x, y), W, yX,
SIMPLIFY = FALSE) Wy \textless- lapply(WX, function(x) x{[}, 1{]}) WX
\textless- lapply(WX, function(x) x{[}, -1, drop = FALSE{]}) A1
\textless- lapply(W, function(x) crossprod(t(crossprod(x, A1)), x)) A1
\textless- Reduce(``+'', A1) minevA1 \textless-
min(eigen(A1)\(values)  eps <- 1e-09  if (minevA1 < eps) {  A1 <- ginv(A1) * length(W)  warning("the first-step matrix is singular, a general inverse is used")  }  else A1 <- solve(A1) * length(W)  WX <- Reduce("+", WX)  Wy <- Reduce("+", Wy)  B1 <- solve(crossprod(WX, t(crossprod(WX, A1))))  Y1 <- crossprod(t(crossprod(WX, A1)), Wy)  coefficients <- as.numeric(crossprod(B1, Y1))  if (effect == "twoways")  names.coef <- c(names.coef, namesV)  names(coefficients) <- names.coef  residuals <- lapply(yX, function(x) as.vector(x[, 1] - crossprod(t(x[,  -1, drop = FALSE]), coefficients)))  outresid <- lapply(residuals, function(x) outer(x, x))  A2 <- mapply(function(x, y) crossprod(t(crossprod(x, y)),  x), W, outresid, SIMPLIFY = FALSE)  A2 <- Reduce("+", A2)  minevA2 <- min(eigen(A2)\)values)
eps \textless- 1e-09 if (minevA2 \textless{} eps) \{ A2 \textless-
ginv(A2) warning(``the second-step matrix is singular, a general inverse
is used'') \} else A2 \textless- solve(A2) B2 \textless-
solve(crossprod(WX, t(crossprod(WX, A2)))) if (model == ``twosteps'') \{
coef1s \textless- coefficients Y2 \textless- crossprod(t(crossprod(WX,
A2)), Wy) coefficients \textless- as.numeric(crossprod(B2, Y2))
names(coefficients) \textless- names.coef vcov \textless- B2 \} else
vcov \textless- B1 rownames(vcov) \textless- colnames(vcov) \textless-
names.coef residuals \textless- lapply(yX, function(x) \{ nz \textless-
rownames(x) z \textless- as.vector(x{[}, 1{]} - crossprod(t(x{[}, -1,
drop = FALSE{]}), coefficients)) names(z) \textless- nz z \})
fitted.values \textless- mapply(function(x, y) x{[}, 1{]} - y, yX,
residuals) if (model == ``twosteps'') coefficients \textless-
list(coef1s, coefficients) args \textless- list(model = model, effect =
effect, transformation = transformation, namest = namesV) result
\textless- list(coefficients = coefficients, residuals = residuals, vcov
= vcov, fitted.values = fitted.values, model = yX, W = W, A1 = A1, A2 =
A2, call = cl, args = args) result \textless- structure(result, class =
c(``pgmm'', ``panelmodel''), pdim = pdim) result \} \textless bytecode:
0x0000000019945dd0\textgreater{} \textless environment:
namespace:plm\textgreater{}

\begin{Shaded}
\begin{Highlighting}[]
\NormalTok{blundell\_bond }\OtherTok{\textless{}{-}} \FunctionTok{pgmm}\NormalTok{(}\AttributeTok{data =}\NormalTok{ dataset2,}
\NormalTok{                          logquantity }\SpecialCharTok{\textasciitilde{}}\NormalTok{ logincome }\SpecialCharTok{+}\NormalTok{ logprice }\SpecialCharTok{+} 
\NormalTok{                            logillegal }\SpecialCharTok{+} \FunctionTok{lag}\NormalTok{(logquantity, }\DecValTok{1}\NormalTok{) }\SpecialCharTok{|} 
                            \FunctionTok{lag}\NormalTok{(logquantity, }\DecValTok{2}\SpecialCharTok{:}\DecValTok{99}\NormalTok{) }\SpecialCharTok{+} \FunctionTok{lag}\NormalTok{(l1\_logquantity, }\DecValTok{2}\SpecialCharTok{:}\DecValTok{99}\NormalTok{))}

\FunctionTok{stargazer}\NormalTok{(blundell\_bond, }\AttributeTok{label =} \StringTok{"tab:reg\_bb"}\NormalTok{, }\AttributeTok{header=}\ConstantTok{FALSE}\NormalTok{, }\AttributeTok{model.names =} \ConstantTok{FALSE}\NormalTok{,}
          \AttributeTok{column.sep.width=}\StringTok{"{-}5pt"}\NormalTok{,}
          \AttributeTok{column.labels =} \FunctionTok{c}\NormalTok{(}\StringTok{\textquotesingle{}Blundell \& Bond\textquotesingle{}}\NormalTok{),}
              \AttributeTok{omit.stat =} \FunctionTok{c}\NormalTok{(}\StringTok{"ll"}\NormalTok{, }\StringTok{"ser"}\NormalTok{, }\StringTok{"rsq"}\NormalTok{))}
\end{Highlighting}
\end{Shaded}

\begin{table}[!htbp] \centering 
  \caption{} 
  \label{tab:reg_bb} 
\begin{tabular}{@{\extracolsep{-5pt}}lc} 
\\[-1.8ex]\hline 
\hline \\[-1.8ex] 
 & \multicolumn{1}{c}{\textit{Dependent variable:}} \\ 
\cline{2-2} 
\\[-1.8ex] & logquantity \\ 
 & Blundell & Bond \\ 
\hline \\[-1.8ex] 
 logincome & 0.677$^{***}$ \\ 
  & (0.235) \\ 
  & \\ 
 logprice & $-$0.434$^{**}$ \\ 
  & (0.174) \\ 
  & \\ 
 logillegal & $-$0.047$^{***}$ \\ 
  & (0.014) \\ 
  & \\ 
 lag(logquantity, 1) & 0.781$^{***}$ \\ 
  & (0.030) \\ 
  & \\ 
\hline \\[-1.8ex] 
Observations & 22 \\ 
\hline 
\hline \\[-1.8ex] 
\textit{Note:}  & \multicolumn{1}{r}{$^{*}$p$<$0.1; $^{**}$p$<$0.05; $^{***}$p$<$0.01} \\ 
\end{tabular} 
\end{table}

\begin{enumerate}
\def\labelenumi{\arabic{enumi}.}
\setcounter{enumi}{7}
\tightlist
\item
  Which parameter estimates do you prefer? Explain why. Are there
  remaining problems with your preferred estimates?
\end{enumerate}

\end{document}
